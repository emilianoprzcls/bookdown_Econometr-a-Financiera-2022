% Options for packages loaded elsewhere
\PassOptionsToPackage{unicode}{hyperref}
\PassOptionsToPackage{hyphens}{url}
%
\documentclass[
]{book}
\usepackage{amsmath,amssymb}
\usepackage{lmodern}
\usepackage{iftex}
\ifPDFTeX
  \usepackage[T1]{fontenc}
  \usepackage[utf8]{inputenc}
  \usepackage{textcomp} % provide euro and other symbols
\else % if luatex or xetex
  \usepackage{unicode-math}
  \defaultfontfeatures{Scale=MatchLowercase}
  \defaultfontfeatures[\rmfamily]{Ligatures=TeX,Scale=1}
\fi
% Use upquote if available, for straight quotes in verbatim environments
\IfFileExists{upquote.sty}{\usepackage{upquote}}{}
\IfFileExists{microtype.sty}{% use microtype if available
  \usepackage[]{microtype}
  \UseMicrotypeSet[protrusion]{basicmath} % disable protrusion for tt fonts
}{}
\makeatletter
\@ifundefined{KOMAClassName}{% if non-KOMA class
  \IfFileExists{parskip.sty}{%
    \usepackage{parskip}
  }{% else
    \setlength{\parindent}{0pt}
    \setlength{\parskip}{6pt plus 2pt minus 1pt}}
}{% if KOMA class
  \KOMAoptions{parskip=half}}
\makeatother
\usepackage{xcolor}
\IfFileExists{xurl.sty}{\usepackage{xurl}}{} % add URL line breaks if available
\IfFileExists{bookmark.sty}{\usepackage{bookmark}}{\usepackage{hyperref}}
\hypersetup{
  pdftitle={06-ESTACIONARIEDAD},
  pdfauthor={Emiliano Pérez Caullieres},
  hidelinks,
  pdfcreator={LaTeX via pandoc}}
\urlstyle{same} % disable monospaced font for URLs
\usepackage{color}
\usepackage{fancyvrb}
\newcommand{\VerbBar}{|}
\newcommand{\VERB}{\Verb[commandchars=\\\{\}]}
\DefineVerbatimEnvironment{Highlighting}{Verbatim}{commandchars=\\\{\}}
% Add ',fontsize=\small' for more characters per line
\usepackage{framed}
\definecolor{shadecolor}{RGB}{248,248,248}
\newenvironment{Shaded}{\begin{snugshade}}{\end{snugshade}}
\newcommand{\AlertTok}[1]{\textcolor[rgb]{0.94,0.16,0.16}{#1}}
\newcommand{\AnnotationTok}[1]{\textcolor[rgb]{0.56,0.35,0.01}{\textbf{\textit{#1}}}}
\newcommand{\AttributeTok}[1]{\textcolor[rgb]{0.77,0.63,0.00}{#1}}
\newcommand{\BaseNTok}[1]{\textcolor[rgb]{0.00,0.00,0.81}{#1}}
\newcommand{\BuiltInTok}[1]{#1}
\newcommand{\CharTok}[1]{\textcolor[rgb]{0.31,0.60,0.02}{#1}}
\newcommand{\CommentTok}[1]{\textcolor[rgb]{0.56,0.35,0.01}{\textit{#1}}}
\newcommand{\CommentVarTok}[1]{\textcolor[rgb]{0.56,0.35,0.01}{\textbf{\textit{#1}}}}
\newcommand{\ConstantTok}[1]{\textcolor[rgb]{0.00,0.00,0.00}{#1}}
\newcommand{\ControlFlowTok}[1]{\textcolor[rgb]{0.13,0.29,0.53}{\textbf{#1}}}
\newcommand{\DataTypeTok}[1]{\textcolor[rgb]{0.13,0.29,0.53}{#1}}
\newcommand{\DecValTok}[1]{\textcolor[rgb]{0.00,0.00,0.81}{#1}}
\newcommand{\DocumentationTok}[1]{\textcolor[rgb]{0.56,0.35,0.01}{\textbf{\textit{#1}}}}
\newcommand{\ErrorTok}[1]{\textcolor[rgb]{0.64,0.00,0.00}{\textbf{#1}}}
\newcommand{\ExtensionTok}[1]{#1}
\newcommand{\FloatTok}[1]{\textcolor[rgb]{0.00,0.00,0.81}{#1}}
\newcommand{\FunctionTok}[1]{\textcolor[rgb]{0.00,0.00,0.00}{#1}}
\newcommand{\ImportTok}[1]{#1}
\newcommand{\InformationTok}[1]{\textcolor[rgb]{0.56,0.35,0.01}{\textbf{\textit{#1}}}}
\newcommand{\KeywordTok}[1]{\textcolor[rgb]{0.13,0.29,0.53}{\textbf{#1}}}
\newcommand{\NormalTok}[1]{#1}
\newcommand{\OperatorTok}[1]{\textcolor[rgb]{0.81,0.36,0.00}{\textbf{#1}}}
\newcommand{\OtherTok}[1]{\textcolor[rgb]{0.56,0.35,0.01}{#1}}
\newcommand{\PreprocessorTok}[1]{\textcolor[rgb]{0.56,0.35,0.01}{\textit{#1}}}
\newcommand{\RegionMarkerTok}[1]{#1}
\newcommand{\SpecialCharTok}[1]{\textcolor[rgb]{0.00,0.00,0.00}{#1}}
\newcommand{\SpecialStringTok}[1]{\textcolor[rgb]{0.31,0.60,0.02}{#1}}
\newcommand{\StringTok}[1]{\textcolor[rgb]{0.31,0.60,0.02}{#1}}
\newcommand{\VariableTok}[1]{\textcolor[rgb]{0.00,0.00,0.00}{#1}}
\newcommand{\VerbatimStringTok}[1]{\textcolor[rgb]{0.31,0.60,0.02}{#1}}
\newcommand{\WarningTok}[1]{\textcolor[rgb]{0.56,0.35,0.01}{\textbf{\textit{#1}}}}
\usepackage{longtable,booktabs,array}
\usepackage{calc} % for calculating minipage widths
% Correct order of tables after \paragraph or \subparagraph
\usepackage{etoolbox}
\makeatletter
\patchcmd\longtable{\par}{\if@noskipsec\mbox{}\fi\par}{}{}
\makeatother
% Allow footnotes in longtable head/foot
\IfFileExists{footnotehyper.sty}{\usepackage{footnotehyper}}{\usepackage{footnote}}
\makesavenoteenv{longtable}
\usepackage{graphicx}
\makeatletter
\def\maxwidth{\ifdim\Gin@nat@width>\linewidth\linewidth\else\Gin@nat@width\fi}
\def\maxheight{\ifdim\Gin@nat@height>\textheight\textheight\else\Gin@nat@height\fi}
\makeatother
% Scale images if necessary, so that they will not overflow the page
% margins by default, and it is still possible to overwrite the defaults
% using explicit options in \includegraphics[width, height, ...]{}
\setkeys{Gin}{width=\maxwidth,height=\maxheight,keepaspectratio}
% Set default figure placement to htbp
\makeatletter
\def\fps@figure{htbp}
\makeatother
\setlength{\emergencystretch}{3em} % prevent overfull lines
\providecommand{\tightlist}{%
  \setlength{\itemsep}{0pt}\setlength{\parskip}{0pt}}
\setcounter{secnumdepth}{5}
\usepackage{booktabs}
\ifLuaTeX
  \usepackage{selnolig}  % disable illegal ligatures
\fi
\usepackage[]{natbib}
\bibliographystyle{apalike}

\title{06-ESTACIONARIEDAD}
\author{Emiliano Pérez Caullieres}
\date{2022-09-21}

\begin{document}
\maketitle

{
\setcounter{tocdepth}{1}
\tableofcontents
}
\hypertarget{muxednimos-cuadrados-ordinarios}{%
\chapter{Mínimos Cuadrados Ordinarios}\label{muxednimos-cuadrados-ordinarios}}

\hypertarget{el-problema}{%
\section{El problema}\label{el-problema}}

Recordando que el método de MCO resulta en encontrar la combinación de valores de los estimadores de los parámetros \(\hat{\boldsymbol{\beta}}\) que permita minimizar la suma de los residuales (estimadores de los términos de erro \(\boldsymbol{\varepsilon}\)) al cuadrado dada por:

\[
    \sum^{N}_{i=1}{e^2_i} = \sum^{N}_{i = 1}{(y_i - \mathbf{X}'_i \hat{\boldsymbol{\beta}})^2}
\]

Donde \(\hat{\boldsymbol{\beta}}\) denota el vector de estimadores \(\hat{\beta}_1, \ldots, \hat{\beta}_K\) y dado que \((e_1, e_2, \ldots, e_n)'(e_1, e_2, \ldots, e_n) = {\mathbf{e'e}}\), el problema del método de MCO consiste en resolver el problema de óptimización:

\begin{eqnarray*}
Minimizar_{\hat{\boldsymbol \beta}} S(\hat{\boldsymbol \beta})  =  Minimizar_{\hat{\boldsymbol \beta}} \mathbf{e'e} \\
    =  Minimizar_{\hat{\boldsymbol \beta}} (\mathbf{Y}-\mathbf{X}\hat{\boldsymbol \beta})'(\mathbf{Y}-\mathbf{X}\hat{\boldsymbol \beta})
\end{eqnarray*}

Expandiendo la expresión \(\mathbf{e'e}\) obtenemos:
\[
    \mathbf{e'e} = \mathbf{Y'Y} - 2 \mathbf{Y'X} \hat{\boldsymbol \beta} + \hat{\boldsymbol \beta}' \mathbf{X'X}\hat{\boldsymbol \beta}
\]

De esta forma obtenemos que las condiciones necesarias de un mínimo son:

\[
    \frac{\partial S(\hat{\boldsymbol \beta})}{\partial \hat{\boldsymbol \beta}} = -2{\mathbf{X'Y}} + 2{\mathbf{X'X}} \hat{\boldsymbol{\beta}} = \mathbf{0}
\]
Y se pueden despejar las \textit{ecuaciones normales} dadas por:

Debido a que el objetivo es encontrar la matriz \(\hat{\boldsymbol\beta}\) despejamos:

\[\hat{\boldsymbol \beta} = (\mathbf{X'X})^{-1}\mathbf{X'Y}
\]
\[
    \mathbf{X'X}\hat{\boldsymbol \beta} = \mathbf{X'Y}
\]

\hypertarget{estimaciuxf3n-r}{%
\section{Estimación R}\label{estimaciuxf3n-r}}

Para la estimación utilizaremos el paquete ``BatchGetSymbols''. Este paquete nos permitirá descargar información acerca de la bolsa de valores internacional.

\hypertarget{dependencias}{%
\subsection{Dependencias}\label{dependencias}}

\begin{Shaded}
\begin{Highlighting}[]
\CommentTok{\#install.packages("pacman")}
\CommentTok{\#pacman nos permite cargar varias librerias en una sola línea}
\FunctionTok{library}\NormalTok{(pacman)}
\NormalTok{pacman}\SpecialCharTok{::}\FunctionTok{p\_load}\NormalTok{(tidyverse,BatchGetSymbols,ggplot2, lubridate)}
\end{Highlighting}
\end{Shaded}

\hypertarget{descarga-de-los-valores}{%
\subsection{Descarga de los valores}\label{descarga-de-los-valores}}

\begin{Shaded}
\begin{Highlighting}[]
\CommentTok{\#Primero determinamos el lapso de tiempo}
\NormalTok{pd}\OtherTok{\textless{}{-}}\FunctionTok{Sys.Date}\NormalTok{()}\SpecialCharTok{{-}}\DecValTok{365} \CommentTok{\#primer fecha}
\NormalTok{pd}
\CommentTok{\#\textgreater{} [1] "2021{-}09{-}21"}
\NormalTok{ld}\OtherTok{\textless{}{-}}\FunctionTok{Sys.Date}\NormalTok{() }\CommentTok{\#última fecha}
\NormalTok{ld}
\CommentTok{\#\textgreater{} [1] "2022{-}09{-}21"}
\CommentTok{\#Intervalos de tiempo}
\NormalTok{int}\OtherTok{\textless{}{-}}\StringTok{"monthly"}
\CommentTok{\#Datos a elegir}
\NormalTok{dt}\OtherTok{\textless{}{-}}\FunctionTok{c}\NormalTok{(}\StringTok{"AMZN"}\NormalTok{)}

\CommentTok{\#Descargando los valores}
\NormalTok{?}\FunctionTok{BatchGetSymbols}\NormalTok{()}
\NormalTok{data}\OtherTok{\textless{}{-}} \FunctionTok{BatchGetSymbols}\NormalTok{(}\AttributeTok{tickers =}\NormalTok{ dt,}
                       \AttributeTok{first.date =}\NormalTok{ pd,}
                       \AttributeTok{last.date =}\NormalTok{ ld,}
                       \AttributeTok{freq.data =}\NormalTok{ int,}
                       \AttributeTok{do.cache =} \ConstantTok{FALSE}\NormalTok{,}
                       \AttributeTok{thresh.bad.data =} \DecValTok{0}\NormalTok{)}

\CommentTok{\#Generando data frame con los valores}
\NormalTok{data\_precio}\OtherTok{\textless{}{-}}\NormalTok{data}\SpecialCharTok{$}\NormalTok{df.tickers}
\FunctionTok{colnames}\NormalTok{(data\_precio)}
\CommentTok{\#\textgreater{}  [1] "ticker"              "ref.date"           }
\CommentTok{\#\textgreater{}  [3] "volume"              "price.open"         }
\CommentTok{\#\textgreater{}  [5] "price.high"          "price.low"          }
\CommentTok{\#\textgreater{}  [7] "price.close"         "price.adjusted"     }
\CommentTok{\#\textgreater{}  [9] "ret.adjusted.prices" "ret.closing.prices"}
\end{Highlighting}
\end{Shaded}

\hypertarget{gruxe1ficas}{%
\subsection{Gráficas}\label{gruxe1ficas}}

\begin{Shaded}
\begin{Highlighting}[]
\NormalTok{sp\_precio}\OtherTok{\textless{}{-}}\FunctionTok{ggplot}\NormalTok{(data\_precio, }\FunctionTok{aes}\NormalTok{(}\AttributeTok{x=}\NormalTok{ref.date, }\AttributeTok{y=}\NormalTok{price.open))}\SpecialCharTok{+}\FunctionTok{geom\_point}\NormalTok{(}\AttributeTok{size =}\DecValTok{2}\NormalTok{, }\AttributeTok{colour =} \StringTok{"black"}\NormalTok{)}\SpecialCharTok{+}\FunctionTok{labs}\NormalTok{(}\AttributeTok{x=}\StringTok{"Fecha"}\NormalTok{, }\AttributeTok{y=}\StringTok{"Precio de apertura (USD)"}\NormalTok{, }\AttributeTok{title=}\StringTok{"Precio de apertura de AMZN en el ultimo año"}\NormalTok{)}\SpecialCharTok{+} \FunctionTok{theme\_light}\NormalTok{()}\SpecialCharTok{+} \FunctionTok{geom\_smooth}\NormalTok{(}\AttributeTok{method =}\NormalTok{ lm, }\AttributeTok{se =} \ConstantTok{TRUE}\NormalTok{)}
\NormalTok{sp\_precio}
\end{Highlighting}
\end{Shaded}

\includegraphics{index_files/figure-latex/unnamed-chunk-3-1.pdf}

\begin{Shaded}
\begin{Highlighting}[]

\NormalTok{sp\_volumen}\OtherTok{\textless{}{-}}\FunctionTok{ggplot}\NormalTok{(data\_precio, }\FunctionTok{aes}\NormalTok{(}\AttributeTok{x=}\NormalTok{ref.date, }\AttributeTok{y=}\NormalTok{volume))}\SpecialCharTok{+}\FunctionTok{geom\_point}\NormalTok{(}\AttributeTok{size =}\DecValTok{2}\NormalTok{, }\AttributeTok{colour =} \StringTok{"black"}\NormalTok{)}\SpecialCharTok{+}\FunctionTok{labs}\NormalTok{(}\AttributeTok{x=}\StringTok{"Fecha"}\NormalTok{, }\AttributeTok{y=}\StringTok{"Volumen"}\NormalTok{, }\AttributeTok{title=}\StringTok{"Volumenes de AMZN en el ultimo año"}\NormalTok{)}\SpecialCharTok{+} \FunctionTok{theme\_light}\NormalTok{()}\SpecialCharTok{+} \FunctionTok{geom\_smooth}\NormalTok{(}\AttributeTok{method =}\NormalTok{ lm, }\AttributeTok{se =} \ConstantTok{TRUE}\NormalTok{)}
\NormalTok{sp\_volumen}
\end{Highlighting}
\end{Shaded}

\includegraphics{index_files/figure-latex/unnamed-chunk-3-2.pdf}

\hypertarget{regresiuxf3n-lineal-que-optiene-los-coeficientes-hatboldsymbol-beta}{%
\subsection{\texorpdfstring{Regresión lineal que optiene los coeficientes \(\hat{\boldsymbol \beta}\)}{Regresión lineal que optiene los coeficientes \textbackslash hat\{\textbackslash boldsymbol \textbackslash beta\}}}\label{regresiuxf3n-lineal-que-optiene-los-coeficientes-hatboldsymbol-beta}}

\begin{Shaded}
\begin{Highlighting}[]
\CommentTok{\#datos estadísticos}
\FunctionTok{summary}\NormalTok{(data\_precio[}\FunctionTok{c}\NormalTok{(}\StringTok{"price.open"}\NormalTok{,}\StringTok{"volume"}\NormalTok{)])}
\CommentTok{\#\textgreater{}    price.open        volume         }
\CommentTok{\#\textgreater{}  Min.   :106.3   Min.   :4.632e+08  }
\CommentTok{\#\textgreater{}  1st Qu.:126.0   1st Qu.:1.273e+09  }
\CommentTok{\#\textgreater{}  Median :152.7   Median :1.465e+09  }
\CommentTok{\#\textgreater{}  Mean   :148.1   Mean   :1.394e+09  }
\CommentTok{\#\textgreater{}  3rd Qu.:167.6   3rd Qu.:1.628e+09  }
\CommentTok{\#\textgreater{}  Max.   :177.2   Max.   :2.258e+09}
\CommentTok{\#análisis de regresión lineal lm() y=precio,x=fecha}
\NormalTok{reg\_tiempo\_precio}\OtherTok{\textless{}{-}}\FunctionTok{lm}\NormalTok{(price.open}\SpecialCharTok{\textasciitilde{}}\NormalTok{ref.date, }\AttributeTok{data=}\NormalTok{data\_precio) }
\CommentTok{\#¡Siempre se pone dentro de lm() la variable dependiente primero y luego la independiete!}
\FunctionTok{summary}\NormalTok{(reg\_tiempo\_precio)}
\CommentTok{\#\textgreater{} }
\CommentTok{\#\textgreater{} Call:}
\CommentTok{\#\textgreater{} lm(formula = price.open \textasciitilde{} ref.date, data = data\_precio)}
\CommentTok{\#\textgreater{} }
\CommentTok{\#\textgreater{} Residuals:}
\CommentTok{\#\textgreater{}      Min       1Q   Median       3Q      Max }
\CommentTok{\#\textgreater{} {-}22.0179  {-}9.1301  {-}0.3498   9.5578  20.7507 }
\CommentTok{\#\textgreater{} }
\CommentTok{\#\textgreater{} Coefficients:}
\CommentTok{\#\textgreater{}               Estimate Std. Error t value Pr(\textgreater{}|t|)    }
\CommentTok{\#\textgreater{} (Intercept) 3308.01329  629.19893   5.257 0.000269 ***}
\CommentTok{\#\textgreater{} ref.date      {-}0.16583    0.03302  {-}5.022 0.000389 ***}
\CommentTok{\#\textgreater{} {-}{-}{-}}
\CommentTok{\#\textgreater{} Signif. codes:  }
\CommentTok{\#\textgreater{} 0 \textquotesingle{}***\textquotesingle{} 0.001 \textquotesingle{}**\textquotesingle{} 0.01 \textquotesingle{}*\textquotesingle{} 0.05 \textquotesingle{}.\textquotesingle{} 0.1 \textquotesingle{} \textquotesingle{} 1}
\CommentTok{\#\textgreater{} }
\CommentTok{\#\textgreater{} Residual standard error: 13.23 on 11 degrees of freedom}
\CommentTok{\#\textgreater{} Multiple R{-}squared:  0.6963, Adjusted R{-}squared:  0.6687 }
\CommentTok{\#\textgreater{} F{-}statistic: 25.22 on 1 and 11 DF,  p{-}value: 0.0003887}

\CommentTok{\#análisis de regresión lineal lm() y=volumen,x=fecha}
\NormalTok{reg\_tiempo\_volumen}\OtherTok{\textless{}{-}}\FunctionTok{lm}\NormalTok{(volume}\SpecialCharTok{\textasciitilde{}}\NormalTok{ref.date, }\AttributeTok{data=}\NormalTok{data\_precio)}
\FunctionTok{summary}\NormalTok{(reg\_tiempo\_volumen)}
\CommentTok{\#\textgreater{} }
\CommentTok{\#\textgreater{} Call:}
\CommentTok{\#\textgreater{} lm(formula = volume \textasciitilde{} ref.date, data = data\_precio)}
\CommentTok{\#\textgreater{} }
\CommentTok{\#\textgreater{} Residuals:}
\CommentTok{\#\textgreater{}        Min         1Q     Median         3Q        Max }
\CommentTok{\#\textgreater{} {-}853866163 {-}111433590   58251296  236217362  837231229 }
\CommentTok{\#\textgreater{} }
\CommentTok{\#\textgreater{} Coefficients:}
\CommentTok{\#\textgreater{}               Estimate Std. Error t value Pr(\textgreater{}|t|)}
\CommentTok{\#\textgreater{} (Intercept) {-}7.512e+09  2.227e+10  {-}0.337    0.742}
\CommentTok{\#\textgreater{} ref.date     4.674e+05  1.169e+06   0.400    0.697}
\CommentTok{\#\textgreater{} }
\CommentTok{\#\textgreater{} Residual standard error: 468200000 on 11 degrees of freedom}
\CommentTok{\#\textgreater{} Multiple R{-}squared:  0.01433,    Adjusted R{-}squared:  {-}0.07527 }
\CommentTok{\#\textgreater{} F{-}statistic:  0.16 on 1 and 11 DF,  p{-}value: 0.6968}
\end{Highlighting}
\end{Shaded}

\hypertarget{ejercicio}{%
\section{Ejercicio}\label{ejercicio}}

El objetivo de este ejrcicio es simplemente que indiquen y modifiquen los errores en el código. Así pues, deberán descomentar \emph{-quitar las \#antes del código-} para empezar el ejercicio.

\hypertarget{section}{%
\subsection{1}\label{section}}

El objetivo de este código es explicar la variable \textbf{``volume''} con la variable \textbf{``price.high''}.

\begin{Shaded}
\begin{Highlighting}[]
\CommentTok{\#reg\_tiempo\_ej1\textless{}{-}lm(price.high\textasciitilde{}volume, data=data\_precio)}
\CommentTok{\#sumary(reg\_tiempo\_ej1)}
\end{Highlighting}
\end{Shaded}

\hypertarget{section-1}{%
\subsection{2}\label{section-1}}

El objetivo de este código es explicar la variable \textbf{``volume''} con la variable \textbf{``price.low''}.

\begin{Shaded}
\begin{Highlighting}[]
\CommentTok{\#reg\_tiempo\_ej2\textless{}{-}lm(price.low\textasciitilde{}volume, data=data\_precio)}
\CommentTok{\#summary(reg\_tiempo\_ej1)}
\end{Highlighting}
\end{Shaded}

\hypertarget{opcional}{%
\subsection{3 (opcional)}\label{opcional}}

El objetivo de este ejercicio es descargar los valores del stock de Tesla \emph{BMV: TSLA} en los últimos \emph{dos años}.

\begin{Shaded}
\begin{Highlighting}[]
\CommentTok{\#dt\_ej3\textless{}{-}("TSLA")}
\CommentTok{\#pdej\textless{}{-}Sys.Date(){-}(365*3) \#primer fecha}
\CommentTok{\#pdej}
\CommentTok{\#Descargando los valores}
\CommentTok{\#dataej3\textless{}{-} BatchgetSymbols(tickers = dt\_ej3,}
                       \CommentTok{\#first.date = pdej,}
                       \CommentTok{\#last.date = ld,}
                       \CommentTok{\#freq.data = int,}
                       \CommentTok{\#do.cache = FALSE,}
                       \CommentTok{\#thresh.bad.data = 0)}

\CommentTok{\#Generando data frame con los valores}
\CommentTok{\#data\_precio\_ej2\textless{}{-}dataej3$df.tickers}
\CommentTok{\#1colnames(data\_precio\_ej2)}
\end{Highlighting}
\end{Shaded}

\hypertarget{muxe1xima-verosimilitud}{%
\chapter{Máxima Verosimilitud}\label{muxe1xima-verosimilitud}}

\hypertarget{el-problema-1}{%
\section{El problema}\label{el-problema-1}}

Recordemos que dado \(f(y_i | \mathbf{x}_i)\) la función de densidad condicional de \(y_i\) dado \(\mathbf{x}_i\). Sea \(\boldsymbol{\theta}\) un conjunto de parámetros de la función. Entonces la función de densidad conjunta de variables aleatorias independientes \(\{ y_i : y_i \in \mathbb{R} \}\) dados los valores \(\{ \mathbf{x}_i : \mathbf{x}_i \in \mathbb{R}^K \}\) estará dada por:

\begin{equation}
    \Pi_{i = 1}^{n} f(y_i | \mathbf{x}_i; \boldsymbol{\theta}) = f(y_1, y_2, \ldots, y_n | \mathbf{x}_1, \mathbf{x}_2, \ldots, \mathbf{x}_n; \boldsymbol{\theta}) = L(\boldsymbol{\theta})
    \label{eq:EqLikehood}
\end{equation}

A la ecuación \eqref{eq:EqLikehood} se le conoce como ecuación de verosimilitud. El problema de máxima verosimilitud entonces será:
\begin{equation}
    \max_{\boldsymbol{\theta} \in \boldsymbol{\Theta}} \Pi_{i = 1}^{n} f(y_i | \mathbf{x}_i; \boldsymbol{\theta}) = \max_{\boldsymbol{\theta} \in \boldsymbol{\Theta}} L(\boldsymbol{\theta})
        \label{eq:EqMaxLike}
\end{equation}

Dado que el logaritmo natural es una transformación monotona, podemos decir que el problema de la ecuación \eqref{eq:EqMaxLike} es equivalente a:

\begin{equation}
     \max_{\boldsymbol{\theta} \in \boldsymbol{\Theta}} ln L(\boldsymbol{\theta}) = \max_{\boldsymbol{\theta} \in \boldsymbol{\Theta}} ln \Pi_{i = 1}^{n} f(y_i | \mathbf{x}_i; \boldsymbol{\theta}) = \max_{\boldsymbol{\theta} \in \boldsymbol{\Theta}} \sum_{i = 1}^{n} ln f(y_i | \mathbf{x}_i; \boldsymbol{\theta})
            \label{eq:EqLogML}
\end{equation}

Para solucionnar el problema se tiene que determinar las condicones de primer y segundo orden, las cuales serán:
\begin{equation}
    \frac{\partial}{\partial \boldsymbol{\theta}} ln L(\boldsymbol{\theta}) = \nabla ln L(\boldsymbol{\theta})
          \label{eq:MLCPO}
\end{equation}

\begin{equation}
    \frac{\partial^2}{\partial^2 \boldsymbol{\theta}} ln L(\boldsymbol{\theta}) = \frac{\partial}{\partial \boldsymbol{\theta}} ln L(\boldsymbol{\theta}) \cdot  \frac{\partial}{\partial \boldsymbol{\theta}} ln L(\boldsymbol{\theta}') = H(\boldsymbol{\theta})
             \label{eq:MLCSO}
\end{equation}

La solución estará dada por aquel valor de \(\hat{\boldsymbol{\theta}}\) que hace:
\begin{equation*}
    \frac{\partial}{\partial \boldsymbol{\theta}} ln L(\hat{\boldsymbol{\theta}}) = 0
\end{equation*}

A su vez, la varianza será aquella que resulta de:
\begin{equation*}
    Var[\hat{\boldsymbol{\theta}} | \mathbf{X}] = \left( - \mathbb{E}_{\hat{\boldsymbol{\theta}}}[H(\boldsymbol{\theta})] \right)^{-1}
\end{equation*}

\hypertarget{estimaciuxf3n-y-simunlaciuxf3n}{%
\section{Estimación y simunlación}\label{estimaciuxf3n-y-simunlaciuxf3n}}

\hypertarget{lanzar-una-moneda}{%
\subsection{Lanzar una moneda}\label{lanzar-una-moneda}}

\begin{Shaded}
\begin{Highlighting}[]
\FunctionTok{set.seed}\NormalTok{(}\DecValTok{1234}\NormalTok{)}\CommentTok{\#esto sirve para siempre generar los mismos numeros aleatorios}
\CommentTok{\#rbinom(numero observaciones,numero de ensayos,probabilidad de exito en cada ensayo)}
\NormalTok{cara}\OtherTok{\textless{}{-}}\FunctionTok{rbinom}\NormalTok{(}\DecValTok{1}\NormalTok{,}\DecValTok{100}\NormalTok{,}\FloatTok{0.5}\NormalTok{)}
\NormalTok{cara}\CommentTok{\#esto nos dice de los 100 ensayos cuantos fueron cara}
\CommentTok{\#\textgreater{} [1] 47}
\NormalTok{sol}\OtherTok{\textless{}{-}}\DecValTok{100}\SpecialCharTok{{-}}\NormalTok{cara}
\NormalTok{sol}
\CommentTok{\#\textgreater{} [1] 53}


\CommentTok{\#Ahora definiremos la función que encontrará la función de verosimilutud para determinado valor p}
\CommentTok{\#}
\NormalTok{verosimilitud }\OtherTok{\textless{}{-}} \ControlFlowTok{function}\NormalTok{(p)\{}
  \FunctionTok{dbinom}\NormalTok{(cara, }\DecValTok{100}\NormalTok{, p)}
\NormalTok{\}}

\CommentTok{\#si suponemos que la probabilidad sesgada de que caiga cara es 40\%}
\NormalTok{prob\_sesgada}\OtherTok{\textless{}{-}}\FloatTok{0.4}
\CommentTok{\#es posible calcular la función de que salga cara}
\FunctionTok{verosimilitud}\NormalTok{(prob\_sesgada)}
\CommentTok{\#\textgreater{} [1] 0.02919091}
\CommentTok{\#ahora es posible generar una función de verimilitud negativa }
\CommentTok{\#para maximizar el valor de la verosimilitud}
\NormalTok{neg\_verosimilitud }\OtherTok{\textless{}{-}} \ControlFlowTok{function}\NormalTok{(p)\{}
  \FunctionTok{dbinom}\NormalTok{(cara, }\DecValTok{100}\NormalTok{, p)}\SpecialCharTok{*{-}}\DecValTok{1}
\NormalTok{\}}
\FunctionTok{neg\_verosimilitud}\NormalTok{(prob\_sesgada)}
\CommentTok{\#\textgreater{} [1] {-}0.02919091}
\CommentTok{\# unamos la función nlm() para maximizar esta función no linear}
\CommentTok{\#?nlm()}
\FunctionTok{nlm}\NormalTok{(neg\_verosimilitud,}\FloatTok{0.5}\NormalTok{,}\AttributeTok{stepmax=}\FloatTok{0.5}\NormalTok{)}\CommentTok{\#se pone un parametro porque sabemos que hay un 0.5 de probabilidad de que caiga cara}
\CommentTok{\#\textgreater{} $minimum}
\CommentTok{\#\textgreater{} [1] {-}0.07973193}
\CommentTok{\#\textgreater{} }
\CommentTok{\#\textgreater{} $estimate}
\CommentTok{\#\textgreater{} [1] 0.47}
\CommentTok{\#\textgreater{} }
\CommentTok{\#\textgreater{} $gradient}
\CommentTok{\#\textgreater{} [1] 1.589701e{-}10}
\CommentTok{\#\textgreater{} }
\CommentTok{\#\textgreater{} $code}
\CommentTok{\#\textgreater{} [1] 1}
\CommentTok{\#\textgreater{} }
\CommentTok{\#\textgreater{} $iterations}
\CommentTok{\#\textgreater{} [1] 4}
\end{Highlighting}
\end{Shaded}

Si bien el ejercicio anterior es un tanto repetitivo debido a que sabemos que hay un 50\% de que caiga una moneda de un lado o otro. Esto ejemplifica la manera en la que se utiliza el metodo de maximización de máxima verosimilitud.

\hypertarget{muxe9todo-generalizado-de-momentos-mgm}{%
\chapter{Método Generalizado de Momentos (MGM)}\label{muxe9todo-generalizado-de-momentos-mgm}}

\hypertarget{el-problema-2}{%
\section{El problema}\label{el-problema-2}}

Retomemos el modelo de regresión lineal tal que:

\begin{equation}
y_i=X_i\beta+u_i
    \label{Eq_reglin}
\end{equation}

Tomando en cuenta los principios de ortogonalidad (\(E(Z_iu_i)=0\)) y (\(rankE(Z_i^{'}X_i)=0\)) sabemos que \(\beta\) es el único vector de \(N\times1\) que resuelve las condiciones de momento de determinada población. En otras palabras, \(E[z_i^{'}(y_i-x_i\beta)]=0\) es una solución y \(E[z_i^{'}(y_i-x_i\beta)]\neq0\) \emph{NO} es una solución. Debido a que la media muestral son estimadores consistentes de momentos de una población, se puede:

\begin{equation}
N^{-1}\sum_{i=1}^{N}z_i^{'}(y_i-x_i\beta)=0
\label{eq:Eqreglin1asolu1}
\end{equation}

Asumiendo que la ecuación \eqref{eq:Eqreglin1asolu1} tiene L ecuaciones lineales y K coeficientes \(\beta\) desconocidos y \(K=L\), entonces la matriz \(\sum_{i=1}^{N}z_i^{'}x_i\) debe ser no singular para encontrar los coeficientes de la siguiente manera.

\begin{equation}
\hat{\beta}=N^{-1}\left[\sum_{i=1}^{N}z_i^{'}x_i\right]^{-1}\left[\sum_{i=1}^{N}z_i^{'}y_i\right]
\label{eq:Eqreglin1asolu2}
\end{equation}

Para simplificar \eqref{eq:Eqreglin1asolu2} se puede nombrar Z juntando \(z_i\) N veces para crear una matriz de tamaño \(NG\times L\). Lo mismo hacemos con X juntando \(x_i\) para obtener una de \(NG\times K\) y Y obteniendo una \(NG\times 1\). Obteniendo:

\begin{equation}
\hat{\beta}=[Z^{'}X]^{-1}[Z^{'}Y]
\end{equation}

Es importante tomar en cuenta cuando el caso en el que hay más ecuaciones lineales que coeficientes \(\beta\); es decir, \(L\geq K\). En estos casos es muy probrable que no haya solución, por lo que mejor que se puede estimar es pones la ecuación \eqref{eq:Eqreglin1asolu1}, tan pequeña como sea posible. Por lo mismo el paso que nos lleva a la ecuación \eqref{eq:Eqreglin1asolu2}, debe eliminarse \(N^{-1}\). El objetivo:

\begin{equation}
\min_{\beta} \left[\sum_{i=1}^{N}z_i^{'}x_i\beta\right]^{-1}\left[\sum_{i=1}^{N}z_i^{'}y_i\beta\right]
\label{eq:Eqreglin1asolu77}
\end{equation}

Así pues nombramos a W como una matriz simétrica de \(W\times W\) donde se genera la variable \(b\) que debemos minimizar que sustituye a \(\beta\) creando una función cuadrática en la ecuación \eqref{eq:Eqreglin1asolu2}.
\begin{equation}
\min_{b}\left[\sum_{i=1}^{N}z_i^{'}x_ib\right]^{-1}\left[\sum_{i=1}^{N}z_i^{'}y_ib\right]
\label{eq:Eqreglin1asolu78}
\end{equation}

\begin{equation}
\therefore\hat{\beta}=[X^{'}Z\hat{W}Z^{'}X]^{-1}[X^{'}Z\hat{W}Z^{'}Y]
\end{equation}

Sin embargo, \(X^{'}Z\hat{W}Z^{'}X\) debe ser no singular para que haya una solución. Para esto se asume que \(\hat{W}\) tiene un limite de probabilidad no singular. Esto se describe como \(\hat{W}\xrightarrow[]{p}W\) y \(N\xrightarrow[]{}W\infty\) donde \(W\) no es aleatorio, es una matriz positiva definida simétrica de \(L\times L\).

\hypertarget{capital-asset-pricing-model-capm}{%
\chapter{CAPITAL ASSET PRICING MODEL (CAPM)}\label{capital-asset-pricing-model-capm}}

\hypertarget{el-problema-3}{%
\section{El problema}\label{el-problema-3}}

Una vez que hemos establecido la manera en la que se pueden estimar algunos valores --como las regresiones lineales y el método de máxima verosimilitud--, además de la naturaleza de los retornos de algunos activos en el capítulo 4, es posible comenzar a hablar de maneras en la que se pueden estimar los valores futuros de los rendimientos de activos y --de esta manera-- poder tomar mejores decisiones de inversiones. Por ello, hablaremos del modelo de \textbf{Capital Asset Pricing Model}. El modelo es muy sencillo y pretende estimar su rentabilidad esperada en función del \textbf{riesgo sistemático}. Por lo mismo, en este modelo se utilizan los valores de los precios de los activos a lo largo del tiempo y utiliza la intuición con la que derivamos la ecuación lineal con los Mínimos cuadrados ordinarios (MCO).

\begin{equation}
    R_{jt}-R{ft}=\alpha_{j}+\beta_j(R_{mt}-R_{ft})+u_{jt}
\label{eq:CAMP}
\end{equation}

En la ecuación \eqref{eq:CAMP}

\begin{itemize}
\item
  \(R_{jt}\) es el retorno del portafolio \(j\) en el tiempo \(t\)
\item
  \(R_{ft}\) es el retorno de un bono sin riesgo gubernamental en un año. \textbf{Parecido a los CETES}.
\item
  \(R_{mt}\) es el retorno en un portafolio de mercado.
\item
  \(u_{jt}\) es el retorno en un portafolio de mercado.
\item
  \(\alpha_{j},\beta_j\) son los coeficientes que queremos obtener.
\end{itemize}

De esta manera, \(\alpha_j\) es el coeficiente que más nos interesa debido a que queremos ver si el activo supera o no el index del mercado con base en el activo fijo.

Si \(\alpha_j\) es positivo entonces sabemos que el retorno tiene buenos rendimiendtos y uno negativo significa que no. Por tanto \(H_0:\alpha_j=0\)

\hypertarget{estimaciuxf3n-r-1}{%
\section{Estimación R}\label{estimaciuxf3n-r-1}}

Para la estimación utilizaremos el paquete ``BatchGetSymbols''. Este paquete nos permitirá descargar información acerca de la bolsa de valores internacional.

\hypertarget{estimaciuxf3n}{%
\section{ESTIMACIÓN}\label{estimaciuxf3n}}

\hypertarget{dependencias-1}{%
\subsection{Dependencias}\label{dependencias-1}}

\begin{Shaded}
\begin{Highlighting}[]
\CommentTok{\#install.packages("pacman")}
\CommentTok{\#pacman nos permite cargar varias librerias en una sola línea}
\FunctionTok{library}\NormalTok{(pacman)}
\NormalTok{pacman}\SpecialCharTok{::}\FunctionTok{p\_load}\NormalTok{(tidyverse,BatchGetSymbols,ggplot2,lubridate,readxl,tidyquant)}
\end{Highlighting}
\end{Shaded}

\hypertarget{descarga-de-los-valores-1}{%
\subsection{Descarga de los valores}\label{descarga-de-los-valores-1}}

\begin{Shaded}
\begin{Highlighting}[]
\CommentTok{\#Primero determinamos el lapso de tiempo}
\NormalTok{pd}\OtherTok{\textless{}{-}}\FunctionTok{as.Date}\NormalTok{(}\StringTok{"2021/09/18"}\NormalTok{) }\CommentTok{\#primer fecha}
\NormalTok{pd}
\CommentTok{\#\textgreater{} [1] "2021{-}09{-}18"}
\NormalTok{ld}\OtherTok{\textless{}{-}}\FunctionTok{as.Date}\NormalTok{(}\StringTok{"2022/09/18"}\NormalTok{) }\CommentTok{\#última fecha}
\NormalTok{ld}
\CommentTok{\#\textgreater{} [1] "2022{-}09{-}18"}
\CommentTok{\#Intervalos de tiempo}
\NormalTok{int}\OtherTok{\textless{}{-}}\StringTok{"monthly"}
\CommentTok{\#Datos a elegir}
\NormalTok{dt}\OtherTok{\textless{}{-}}\FunctionTok{c}\NormalTok{(}\StringTok{"AMZN"}\NormalTok{)}
\NormalTok{dt2}\OtherTok{\textless{}{-}}\FunctionTok{c}\NormalTok{(}\StringTok{"TSLA"}\NormalTok{)}
\CommentTok{\#Descargando los valores}
\NormalTok{?}\FunctionTok{BatchGetSymbols}\NormalTok{()}
\NormalTok{data1}\OtherTok{\textless{}{-}} \FunctionTok{BatchGetSymbols}\NormalTok{(}\AttributeTok{tickers =}\NormalTok{ dt,}
                       \AttributeTok{first.date =}\NormalTok{ pd,}
                       \AttributeTok{last.date =}\NormalTok{ ld,}
                       \AttributeTok{freq.data =}\NormalTok{ int,}
                       \AttributeTok{do.cache =} \ConstantTok{FALSE}\NormalTok{,}
                       \AttributeTok{thresh.bad.data =} \DecValTok{0}\NormalTok{)}
\NormalTok{data2}\OtherTok{\textless{}{-}} \FunctionTok{BatchGetSymbols}\NormalTok{(}\AttributeTok{tickers =}\NormalTok{ dt2,}
                       \AttributeTok{first.date =}\NormalTok{ pd,}
                       \AttributeTok{last.date =}\NormalTok{ ld,}
                       \AttributeTok{freq.data =}\NormalTok{ int,}
                       \AttributeTok{do.cache =} \ConstantTok{FALSE}\NormalTok{,}
                       \AttributeTok{thresh.bad.data =} \DecValTok{0}\NormalTok{)}

\CommentTok{\#Generando data frame con los valores}
\NormalTok{data\_precio\_amzn}\OtherTok{\textless{}{-}}\NormalTok{data1}\SpecialCharTok{$}\NormalTok{df.tickers}
\FunctionTok{colnames}\NormalTok{(data\_precio\_amzn)}
\CommentTok{\#\textgreater{}  [1] "ticker"              "ref.date"           }
\CommentTok{\#\textgreater{}  [3] "volume"              "price.open"         }
\CommentTok{\#\textgreater{}  [5] "price.high"          "price.low"          }
\CommentTok{\#\textgreater{}  [7] "price.close"         "price.adjusted"     }
\CommentTok{\#\textgreater{}  [9] "ret.adjusted.prices" "ret.closing.prices"}
\NormalTok{data\_precio\_tls}\OtherTok{\textless{}{-}}\NormalTok{data2}\SpecialCharTok{$}\NormalTok{df.tickers}
\FunctionTok{colnames}\NormalTok{(data\_precio\_tls)}
\CommentTok{\#\textgreater{}  [1] "ticker"              "ref.date"           }
\CommentTok{\#\textgreater{}  [3] "volume"              "price.open"         }
\CommentTok{\#\textgreater{}  [5] "price.high"          "price.low"          }
\CommentTok{\#\textgreater{}  [7] "price.close"         "price.adjusted"     }
\CommentTok{\#\textgreater{}  [9] "ret.adjusted.prices" "ret.closing.prices"}
\CommentTok{\#necesitamos convertir la serie de tiempo de precios en retornos continuos compuestos de los precios de apertura}
\NormalTok{data\_precio\_amzn}\SpecialCharTok{$}\NormalTok{ccrAMZN}\OtherTok{\textless{}{-}}\FunctionTok{c}\NormalTok{(}\ConstantTok{NA}\NormalTok{ ,}\DecValTok{100}\SpecialCharTok{*}\FunctionTok{diff}\NormalTok{(}\FunctionTok{log}\NormalTok{(data\_precio\_amzn}\SpecialCharTok{$}\NormalTok{price.open)))}\CommentTok{\#agregamos un valor NA al principio}
\NormalTok{data\_precio\_amzn}\SpecialCharTok{$}\NormalTok{ccrAMZN}\CommentTok{\#estos son los retornos}
\CommentTok{\#\textgreater{}  [1]          NA  {-}3.2011678   2.1889913   5.3061639}
\CommentTok{\#\textgreater{}  [5]  {-}5.6279333 {-}11.0646538   1.8052718   7.2089622}
\CommentTok{\#\textgreater{}  [9] {-}29.3475086  {-}0.1185366 {-}13.9945965  23.8807273}
\CommentTok{\#\textgreater{} [13]  {-}6.8696583}

\NormalTok{data\_precio\_tls}\SpecialCharTok{$}\NormalTok{ccrTSLA}\OtherTok{\textless{}{-}}\FunctionTok{c}\NormalTok{(}\ConstantTok{NA}\NormalTok{ ,}\DecValTok{100}\SpecialCharTok{*}\FunctionTok{diff}\NormalTok{(}\FunctionTok{log}\NormalTok{(data\_precio\_tls}\SpecialCharTok{$}\NormalTok{price.open)))}\CommentTok{\#agregamos un valor NA al principio}
\NormalTok{data\_precio\_tls}\SpecialCharTok{$}\NormalTok{ccrTSLA}
\CommentTok{\#\textgreater{}  [1]         NA   5.796888  38.591933   1.361865  {-}1.121972}
\CommentTok{\#\textgreater{}  [6] {-}20.478772  {-}7.264574  21.765521 {-}22.795320 {-}13.089771}
\CommentTok{\#\textgreater{} [11] {-}10.336735  28.307900 {-}10.009692}
\CommentTok{\#formateando por año y mes}
\NormalTok{data\_precio\_tls}\SpecialCharTok{$}\NormalTok{ref.date}\OtherTok{=}\FunctionTok{format}\NormalTok{(}\FunctionTok{as.Date}\NormalTok{(data\_precio\_tls}\SpecialCharTok{$}\NormalTok{ref.date), }\StringTok{"\%m/\%Y"}\NormalTok{)}
\NormalTok{data\_precio\_amzn}\SpecialCharTok{$}\NormalTok{ref.date}\OtherTok{=}\FunctionTok{format}\NormalTok{(}\FunctionTok{as.Date}\NormalTok{(data\_precio\_amzn}\SpecialCharTok{$}\NormalTok{ref.date), }\StringTok{"\%m/\%Y"}\NormalTok{)}
\CommentTok{\#Compararemos con los CETES}
\NormalTok{CETES\_sep2021\_2022}\OtherTok{\textless{}{-}}\FunctionTok{read\_excel}\NormalTok{(}\StringTok{"BD/CETES{-}sep2021{-}2022.xlsx"}\NormalTok{, }\AttributeTok{skip=}\DecValTok{17}\NormalTok{)}
\FunctionTok{head}\NormalTok{(CETES\_sep2021\_2022)}
\CommentTok{\#\textgreater{} \# A tibble: 6 x 2}
\CommentTok{\#\textgreater{}   Fecha               SF43936}
\CommentTok{\#\textgreater{}   \textless{}dttm\textgreater{}                \textless{}dbl\textgreater{}}
\CommentTok{\#\textgreater{} 1 2021{-}09{-}15 00:00:00    4.6 }
\CommentTok{\#\textgreater{} 2 2021{-}09{-}23 00:00:00    4.58}
\CommentTok{\#\textgreater{} 3 2021{-}09{-}30 00:00:00    4.69}
\CommentTok{\#\textgreater{} 4 2021{-}10{-}07 00:00:00    4.81}
\CommentTok{\#\textgreater{} 5 2021{-}10{-}14 00:00:00    4.79}
\CommentTok{\#\textgreater{} 6 2021{-}10{-}21 00:00:00    4.83}
\CommentTok{\#indice sp500}
\NormalTok{SP500 }\OtherTok{\textless{}{-}} \FunctionTok{read\_csv}\NormalTok{(}\StringTok{"BD/Download Data {-} INDEX\_US\_S\&P US\_SPX.csv"}\NormalTok{)}
\NormalTok{SP500}\SpecialCharTok{$}\NormalTok{ccrSP500}\OtherTok{\textless{}{-}}\FunctionTok{c}\NormalTok{(}\ConstantTok{NA}\NormalTok{ ,}\DecValTok{100}\SpecialCharTok{*}\FunctionTok{diff}\NormalTok{(}\FunctionTok{log}\NormalTok{(SP500}\SpecialCharTok{$}\NormalTok{Open)))}
\FunctionTok{names}\NormalTok{(SP500)[}\DecValTok{1}\NormalTok{]}\OtherTok{\textless{}{-}}\FunctionTok{paste}\NormalTok{(}\StringTok{\textquotesingle{}ref.date\textquotesingle{}}\NormalTok{)}
\CommentTok{\#formateando por año y mes}

\CommentTok{\#cetes}
\NormalTok{cete\_1\_año}\OtherTok{\textless{}{-}}\FloatTok{10.10}\CommentTok{\#esto es el rendimiento a un año de un cete gubernamental seguro}

\CommentTok{\#Juntamos el df}
\NormalTok{CAPM\_2}\OtherTok{\textless{}{-}}\FunctionTok{merge}\NormalTok{(data\_precio\_amzn, data\_precio\_tls, }\AttributeTok{by =} \FunctionTok{c}\NormalTok{(}\StringTok{\textquotesingle{}ref.date\textquotesingle{}}\NormalTok{))}
\NormalTok{CAPM\_4}\OtherTok{\textless{}{-}}\FunctionTok{merge}\NormalTok{(SP500, CAPM\_2, }\AttributeTok{by =} \FunctionTok{c}\NormalTok{(}\StringTok{\textquotesingle{}ref.date\textquotesingle{}}\NormalTok{))}
\NormalTok{CAPM}\OtherTok{\textless{}{-}}\FunctionTok{data.frame}\NormalTok{(CAPM\_4)}

\CommentTok{\#exceso de retorno}
\NormalTok{CAPM}\SpecialCharTok{$}\NormalTok{excess\_ret\_AMZN}\OtherTok{\textless{}{-}}\NormalTok{CAPM}\SpecialCharTok{$}\NormalTok{ccrAMZN}\SpecialCharTok{{-}}\NormalTok{cete\_1\_año}
\NormalTok{CAPM}\SpecialCharTok{$}\NormalTok{excess\_ret\_SP500}\OtherTok{\textless{}{-}}\NormalTok{CAPM}\SpecialCharTok{$}\NormalTok{ccrSP500}\SpecialCharTok{{-}}\NormalTok{cete\_1\_año}
\NormalTok{CAPM}\SpecialCharTok{$}\NormalTok{excess\_ret\_TSLA}\OtherTok{\textless{}{-}}\NormalTok{CAPM}\SpecialCharTok{$}\NormalTok{ccrTSLA}\SpecialCharTok{{-}}\NormalTok{cete\_1\_año}
\end{Highlighting}
\end{Shaded}

\begin{Shaded}
\begin{Highlighting}[]
\CommentTok{\#relacion entre los excesos de demanda}
\FunctionTok{ggplot}\NormalTok{(CAPM, }\FunctionTok{aes}\NormalTok{(}\AttributeTok{x=}\NormalTok{excess\_ret\_AMZN, }\AttributeTok{y=}\NormalTok{excess\_ret\_SP500))}\SpecialCharTok{+}\FunctionTok{geom\_point}\NormalTok{()}\SpecialCharTok{+}\FunctionTok{labs}\NormalTok{(}\AttributeTok{title=}\StringTok{"Relación de excesos de retornos entre TSLA y AMZN"}\NormalTok{,}\AttributeTok{y=}\StringTok{"Exceso de demanda de SP500"}\NormalTok{, }\AttributeTok{x=}\StringTok{"Exceso de demanda de AMZN"}\NormalTok{)}\SpecialCharTok{+}\FunctionTok{theme\_light}\NormalTok{()}
\CommentTok{\#\textgreater{} Warning: Removed 2 rows containing missing values}
\CommentTok{\#\textgreater{} (geom\_point).}
\end{Highlighting}
\end{Shaded}

\begin{figure}
\centering
\includegraphics{04-CAPM_files/figure-latex/TSTSLAAMAZN1-1.pdf}
\caption{\label{fig:TSTSLAAMAZN1}Relación de excesos de retornos entre AMZN y SP500}
\end{figure}

\begin{Shaded}
\begin{Highlighting}[]
\CommentTok{\#relacion entre los excesos de demanda}
\FunctionTok{ggplot}\NormalTok{(CAPM, }\FunctionTok{aes}\NormalTok{(}\AttributeTok{x=}\NormalTok{excess\_ret\_TSLA, }\AttributeTok{y=}\NormalTok{excess\_ret\_SP500))}\SpecialCharTok{+}\FunctionTok{geom\_point}\NormalTok{()}\SpecialCharTok{+}\FunctionTok{labs}\NormalTok{(}\AttributeTok{title=}\StringTok{"Relación de excesos de retornos entre TSLA y AMZN"}\NormalTok{,}\AttributeTok{y=}\StringTok{"Exceso de demanda de SP500"}\NormalTok{, }\AttributeTok{x=}\StringTok{"Exceso de demanda de TSLA"}\NormalTok{)}\SpecialCharTok{+}\FunctionTok{theme\_light}\NormalTok{()}
\CommentTok{\#\textgreater{} Warning: Removed 2 rows containing missing values}
\CommentTok{\#\textgreater{} (geom\_point).}
\end{Highlighting}
\end{Shaded}

\begin{figure}
\centering
\includegraphics{04-CAPM_files/figure-latex/TSTSLAAMAZN2-1.pdf}
\caption{\label{fig:TSTSLAAMAZN2}Relación de excesos de retornos entre TSLA y SP500}
\end{figure}

\begin{Shaded}
\begin{Highlighting}[]
\CommentTok{\#veamos la regresion lineal}
\NormalTok{CAPM\_lr}\OtherTok{\textless{}{-}}\FunctionTok{lm}\NormalTok{(excess\_ret\_TSLA}\SpecialCharTok{\textasciitilde{}}\NormalTok{excess\_ret\_SP500,}\AttributeTok{data =}\NormalTok{ CAPM)}
\FunctionTok{summary}\NormalTok{(CAPM\_lr)}
\CommentTok{\#\textgreater{} }
\CommentTok{\#\textgreater{} Call:}
\CommentTok{\#\textgreater{} lm(formula = excess\_ret\_TSLA \textasciitilde{} excess\_ret\_SP500, data = CAPM)}
\CommentTok{\#\textgreater{} }
\CommentTok{\#\textgreater{} Residuals:}
\CommentTok{\#\textgreater{}     Min      1Q  Median      3Q     Max }
\CommentTok{\#\textgreater{} {-}23.980 {-}13.391  {-}5.548  11.572  37.067 }
\CommentTok{\#\textgreater{} }
\CommentTok{\#\textgreater{} Coefficients:}
\CommentTok{\#\textgreater{}                  Estimate Std. Error t value Pr(\textgreater{}|t|)}
\CommentTok{\#\textgreater{} (Intercept)       {-}3.2334    11.8097  {-}0.274     0.79}
\CommentTok{\#\textgreater{} excess\_ret\_SP500   0.5379     1.0789   0.499     0.63}
\CommentTok{\#\textgreater{} }
\CommentTok{\#\textgreater{} Residual standard error: 20.88 on 9 degrees of freedom}
\CommentTok{\#\textgreater{}   (2 observations deleted due to missingness)}
\CommentTok{\#\textgreater{} Multiple R{-}squared:  0.02687,    Adjusted R{-}squared:  {-}0.08125 }
\CommentTok{\#\textgreater{} F{-}statistic: 0.2486 on 1 and 9 DF,  p{-}value: 0.6301}
\NormalTok{alpha1}\OtherTok{\textless{}{-}}\FunctionTok{coefficients}\NormalTok{(CAPM\_lr)[}\DecValTok{1}\NormalTok{]}
\NormalTok{alpha1}\SpecialCharTok{\textless{}}\DecValTok{0}
\CommentTok{\#\textgreater{} (Intercept) }
\CommentTok{\#\textgreater{}        TRUE}
\end{Highlighting}
\end{Shaded}

De esta manera sabemos que el rendimiento de TSLA NO es mayor debido a que el coeficiente \(\alpha=-2.9534\), lo cual indica peores rendimientos al resto del SP500.

\hypertarget{ejercicio-compara-con-tsla-con-el-apple}{%
\section{Ejercicio Compara con TSLA con el APPLE}\label{ejercicio-compara-con-tsla-con-el-apple}}

\begin{Shaded}
\begin{Highlighting}[]
\NormalTok{dt3}\OtherTok{\textless{}{-}}\StringTok{"AAPL"}
\NormalTok{data3}\OtherTok{\textless{}{-}}\FunctionTok{BatchGetSymbols}\NormalTok{(}\AttributeTok{tickers =}\NormalTok{ dt3,}
                       \AttributeTok{first.date =}\NormalTok{ pd,}
                       \AttributeTok{last.date =}\NormalTok{ ld,}
                       \AttributeTok{freq.data =}\NormalTok{ int,}
                       \AttributeTok{do.cache =} \ConstantTok{FALSE}\NormalTok{,}
                       \AttributeTok{thresh.bad.data =} \DecValTok{0}\NormalTok{)}
\CommentTok{\#\textgreater{} Warning: \textasciigrave{}BatchGetSymbols()\textasciigrave{} was deprecated in BatchGetSymbols 2.6.4.}
\CommentTok{\#\textgreater{} Please use \textasciigrave{}yfR::yf\_get()\textasciigrave{} instead.}
\CommentTok{\#\textgreater{} 2022{-}05{-}01: Package BatchGetSymbols will soon be replaced by yfR. }
\CommentTok{\#\textgreater{} More details about the change is available at github \textless{}\textless{}www.github.com/msperlin/yfR\textgreater{}}
\CommentTok{\#\textgreater{} You can install yfR by executing:}
\CommentTok{\#\textgreater{} }
\CommentTok{\#\textgreater{} remotes::install\_github(\textquotesingle{}msperlin/yfR\textquotesingle{})}
\CommentTok{\#\textgreater{} }
\CommentTok{\#\textgreater{} Running BatchGetSymbols for:}
\CommentTok{\#\textgreater{}    tickers =AAPL}
\CommentTok{\#\textgreater{}    Downloading data for benchmark ticker}
\CommentTok{\#\textgreater{} \^{}GSPC | yahoo (1|1)}
\CommentTok{\#\textgreater{} AAPL | yahoo (1|1) {-} Got 100\% of valid prices | Youre doing good!}
\NormalTok{data\_precio\_AAPL}\OtherTok{\textless{}{-}}\NormalTok{data3}\SpecialCharTok{$}\NormalTok{df.tickers}
\FunctionTok{colnames}\NormalTok{(data\_precio\_AAPL)}
\CommentTok{\#\textgreater{}  [1] "ticker"              "ref.date"           }
\CommentTok{\#\textgreater{}  [3] "volume"              "price.open"         }
\CommentTok{\#\textgreater{}  [5] "price.high"          "price.low"          }
\CommentTok{\#\textgreater{}  [7] "price.close"         "price.adjusted"     }
\CommentTok{\#\textgreater{}  [9] "ret.adjusted.prices" "ret.closing.prices"}
\NormalTok{data\_precio\_AAPL}\SpecialCharTok{$}\NormalTok{ccrAAPL}\OtherTok{\textless{}{-}}\FunctionTok{c}\NormalTok{(}\ConstantTok{NA}\NormalTok{ ,}\DecValTok{100}\SpecialCharTok{*}\FunctionTok{diff}\NormalTok{(}\FunctionTok{log}\NormalTok{(data\_precio\_AAPL}\SpecialCharTok{$}\NormalTok{price.open)))}\CommentTok{\#agregamos un valor NA al principio}
\NormalTok{data\_precio\_AAPL}\SpecialCharTok{$}\NormalTok{ccrAAPL}
\CommentTok{\#\textgreater{}  [1]         NA  {-}1.330092   4.875668  11.698469   5.996413}
\CommentTok{\#\textgreater{}  [6]  {-}2.171531  {-}5.498712   5.510207 {-}10.483068  {-}4.442864}
\CommentTok{\#\textgreater{} [11]  {-}9.701946  16.851754  {-}2.751627}
\NormalTok{data\_precio\_AAPL}\SpecialCharTok{$}\NormalTok{ref.date}\OtherTok{=}\FunctionTok{format}\NormalTok{(}\FunctionTok{as.Date}\NormalTok{(data\_precio\_AAPL}\SpecialCharTok{$}\NormalTok{ref.date), }\StringTok{"\%m/\%Y"}\NormalTok{)}
\NormalTok{CAPM\_3}\OtherTok{\textless{}{-}}\FunctionTok{merge}\NormalTok{(data\_precio\_AAPL, CAPM, }\AttributeTok{by =} \FunctionTok{c}\NormalTok{(}\StringTok{\textquotesingle{}ref.date\textquotesingle{}}\NormalTok{))}
\NormalTok{CAPM\_3}\SpecialCharTok{$}\NormalTok{excess\_ret\_AAPL}\OtherTok{\textless{}{-}}\NormalTok{CAPM\_3}\SpecialCharTok{$}\NormalTok{ccrAAPL}\SpecialCharTok{{-}}\NormalTok{cete\_1\_año}
\CommentTok{\#veamos la regresion lineal}
\NormalTok{CAPM3\_lr}\OtherTok{\textless{}{-}}\FunctionTok{lm}\NormalTok{(excess\_ret\_AAPL}\SpecialCharTok{\textasciitilde{}}\NormalTok{excess\_ret\_SP500,}\AttributeTok{data =}\NormalTok{ CAPM\_3)}
\FunctionTok{summary}\NormalTok{(CAPM3\_lr)}
\CommentTok{\#\textgreater{} }
\CommentTok{\#\textgreater{} Call:}
\CommentTok{\#\textgreater{} lm(formula = excess\_ret\_AAPL \textasciitilde{} excess\_ret\_SP500, data = CAPM\_3)}
\CommentTok{\#\textgreater{} }
\CommentTok{\#\textgreater{} Residuals:}
\CommentTok{\#\textgreater{}      Min       1Q   Median       3Q      Max }
\CommentTok{\#\textgreater{} {-}10.9038  {-}5.4365   0.4641   3.4629  14.1798 }
\CommentTok{\#\textgreater{} }
\CommentTok{\#\textgreater{} Coefficients:}
\CommentTok{\#\textgreater{}                  Estimate Std. Error t value Pr(\textgreater{}|t|)}
\CommentTok{\#\textgreater{} (Intercept)       {-}4.7542     4.9105  {-}0.968    0.358}
\CommentTok{\#\textgreater{} excess\_ret\_SP500   0.4663     0.4486   1.039    0.326}
\CommentTok{\#\textgreater{} }
\CommentTok{\#\textgreater{} Residual standard error: 8.682 on 9 degrees of freedom}
\CommentTok{\#\textgreater{}   (2 observations deleted due to missingness)}
\CommentTok{\#\textgreater{} Multiple R{-}squared:  0.1072, Adjusted R{-}squared:  0.007964 }
\CommentTok{\#\textgreater{} F{-}statistic:  1.08 on 1 and 9 DF,  p{-}value: 0.3258}
\NormalTok{alpha2}\OtherTok{\textless{}{-}}\FunctionTok{coefficients}\NormalTok{(CAPM3\_lr)[}\DecValTok{1}\NormalTok{]}
\NormalTok{alpha2}\SpecialCharTok{\textless{}}\DecValTok{0}
\CommentTok{\#\textgreater{} (Intercept) }
\CommentTok{\#\textgreater{}        TRUE}
\end{Highlighting}
\end{Shaded}

\hypertarget{estacionariedad}{%
\chapter{Estacionariedad}\label{estacionariedad}}

\hypertarget{el-problema-4}{%
\section{El problema}\label{el-problema-4}}

Los fundamentos de las series de tiempo están basados en la
estacionalidad. Una serie de tiempo \({r_t}\) que estudia los retornos de
un activo a lo largo de tiempo es \emph{estrictamente estacionaria} si la
distribución conjunta de los retornos \((r_{t1},\dots,r_{t1})\) es
\emph{exactamente idéntica} en \((r_{t1+T},\dots,r_{t1+T})\), es decir cuando
pasa \(T\) años, por ejemplo. En otras palabras, definiremos a una serie
de tiempo como un vector de variables \({X_t}\) aleatorias de dimensión
\(T\), dado como:

\begin{equation}
    X_1, X_2, X_3, \ldots ,X_T
\end{equation}

Es decir, definiremos a una serie de tiempo como una
realización de un proceso estocástico --o un Proceso Generador de Datos
(PGD). Consideremos una muestra de los múltiples posibles resultados de
muestras de tamaño \(T\), la colección dada por:

\begin{equation}
    \{X^{(1)}_1, X^{(1)}_2, \ldots, X^{(1)}_T\}
    \label{eq:variast}
\end{equation}

Eventualmente podríamos estar dispuestos a observar este proceso
indefinidamente, de forma tal que estemos interesados en observar a la
secuencia dada por \(\{ X^{(1)}_t \}^{\infty}_{t = 1}\), lo cual no
dejaría se ser sólo una de las tantas realizaciones o secuencias del
proceso estocástico original de la ecuación \eqref{eq:variast}.

\begin{eqnarray*}
    & \{X^{(2)}_1, X^{(2)}_2, \ldots, X^{(2)}_T\} & \\
    & \{X^{(3)}_1, X^{(3)}_2, \ldots, X^{(3)}_T\} & \\
    & \{X^{(4)}_1, X^{(4)}_2, \ldots, X^{(4)}_T\} & \\
    & \vdots & \\
    & \{X^{(j)}_1, X^{(j)}_2, \ldots, X^{(j)}_T\} & 
\end{eqnarray*}

Por lo mismo, cada cambio que se hace al vector \(\{ X^{(1)}_t \}\) es
parte del mismo proceso estocástico, por lo que la serie de tiempo es:

\begin{equation}
    \{ X_1, X_2, \ldots, X_T \}
        \label{variast}
\end{equation}

El proceso estocástico de dimensión \(T\) puede ser completamente descrito
por su función de distribución multivariada de dimensión \(T\). No
obstante, sólo nos enfocaremos en sus primer y segundo momentos, es
decir, en sus medias o valores esperados \(\mathbb{E} (X_t)\)

Para \(t = 1, 2, \ldots, T\):

\begin{equation*}
\left[
    \begin{array}{c}
    \mathbb{E}[X_1], \mathbb{E}[X_2], \ldots, \mathbb{E}[X_T]
    \end{array}
\right]
\end{equation*}

De sus variazas:

\begin{equation*}
    Var[X_t] = \mathbb{E}[(X_t - \mathbb{E}[X_t])^2]
\end{equation*} Para \(t = 1, 2, \ldots, T\), y de sus \(T(T-1)/2\)
covarianzas: \begin{equation*}
    Cov[X_t,X_s] = \mathbb{E}[(X_t - \mathbb{E}[X_t])(X_s - \mathbb{E}[X_s])]
\end{equation*}

Para \(t < s\). Por lo tanto, en la forma matricial podemos escribir lo siguiente:
\begin{equation*}
\left[
    \begin{array}{c c c c}
    Var[X_1] & Cov[X_1,X_2] & \cdots & Cov[X_1,X_T] \\
    Cov[X_2,X_1] & Var[X_2] & \cdots & Cov[X_2,X_T] \\
    \vdots & \vdots & \ddots & \vdots \\
    Cov[X_T,X_1] & Cov[X_T,X_2] & \cdots & Var[X_T] \\
    \end{array}
\right]
\end{equation*}

\begin{equation}
= \left[
    \begin{array}{c c c c}
    \sigma_1^2 & \rho_{12} & \cdots & \rho_{1T} \\
    \rho_{21} & \sigma_2^2 & \cdots & \rho_{2T} \\
    \vdots & \vdots & \ddots & \vdots \\
    \rho_{T1} & \rho_{T2} & \cdots & \sigma_T^2 \\
    \end{array}
\right]
\label{eq:MATCOV}
\end{equation}

Donde es claro que en la matriz de la ecuación \eqref{eq:MATCOV} existen \(T(T-1)/2\) covarianzas distintas, ya que se cumple que \(Cov[X_t,X_s] = Cov[X_s,X_t]\), para \(t \neq s\). A menudo, esas covarianzas son denominadas como \emph{autocovarianzas} puesto que ellas son covarianzas entre variables aleatorias pertenecientes al mismo proceso estocástico pero en un momento \(t\) diferente. Si el proceso estocástico tiene una distribución normal multivariada, su función de distribución estará totalmente descrita por sus momentos de primer y segundo orden.

\hypertarget{ergocidad}{%
\subsection{Ergocidad}\label{ergocidad}}

Esto implica que los momentos muestrales, los cuales son calculados en la base de una serie de tiempo con un número finito de observaciones, conforme el tiempo \(T \rightarrow \infty\) sus correspondientes momentos muestrales, tienden a los verdaderos valores poblacionales, los cuales definiremos como \(\mu\), para la media, y \(\sigma^2_X\) para la varianza. \emph{En pocas palabras, conforme los momentos muestrales aumenten tanto que tiendan al infinito, entonces nos acercamos a valores poblacionales de la media y la varianza}.
Este concepto sólo es cierto si asumimos que

\begin{eqnarray*}
    \mathbb{E}[X_t] = \mu_t = \mu \\
    Var[X_t] = \sigma^2_X
\end{eqnarray*}
Más formalmente, se dice que el PGD o el proceso estocástico es ergódico en la media si:
\begin{equation}
    \displaystyle\lim_{T \to \infty}{\mathbb{E} \left[ \left( \frac{1}{T} \sum^{T}_{t = 1} (X_t - \mu) \right) ^2 \right]} = 0
\end{equation}

y ergódico en la varianza si:
\begin{equation}
    \displaystyle\lim_{T \to \infty}{\mathbb{E} \left[ \left( \frac{1}{T} \sum^{T}_{t = 1} (X_t - \mu) ^2 - \sigma^2_X \right) ^2 \right]} = 0
\end{equation}
+
Estas condiciones se les conoce como \emph{propiedades de consistencia} para las variables aleatorias. Sin embargo, éstas no pueden ser probadas. Por ello se les denomina como un supuesto que pueden cumplir algunas de las series. Más importante aún: \textbf{un proceso estocástico que tiende a estar en equilibrio estadístico en un orden ergódico, es estacionario}.

\hypertarget{tipos-de-estacionariedad}{%
\subsection{Tipos de Estacionariedad}\label{tipos-de-estacionariedad}}

Definiremos a la estacionariedad por sus momentos del correspondiente proceso estocástico dado por \(\{X_t\}\):

\begin{itemize}
\item
  \emph{Estacionariedad en media}: Un proceso estocástico es estacionario en media si \(E[X_t] = \mu_t = \mu\) es constante para todo \(t\).
\item
  \emph{Estacionariedad en varianza}: Un proceso estocástico es estacionario en varianza si \(Var[X_t] = \mathbb{E}[(X_t - \mu_t)^2] = \sigma^2_X = \gamma(0)\) es constante y finita para todo \(t\).
\item
  \emph{Estacionariedad en covarianza}: Un proceso estocástico es estacionario en covarianza si \(Cov[X_t,X_s] = \mathbb{E}[(X_t - \mu_t)(X_s - \mu_s)] = \gamma(|s-t|)\) es sólo una función del tiempo y de la distancia entre las dos variables aleatorias. Por lo que no depende del tiempo denotado por \(t\) (no depende de la información contemporánea).
\item
  \emph{Estacionariedad débil}: Como la estacionariedad en varianza resulta de forma inmediata de la estacionariedad en covarianza cuando se asume que \(s = t\), un proceso estocástico es débilmente estacionario cuando es estacionario en media y covarianza. \textbf{ESTE ES EL MÁS COMÚN Y POSIBLE}, por lo que es el que estudiaremos.
\end{itemize}

\hypertarget{funciuxf3n-de-autocorrelaciuxf3n-acf}{%
\subsection{Función de Autocorrelación (ACF)}\label{funciuxf3n-de-autocorrelaciuxf3n-acf}}

Para ampliar la discusión, es posible calcular la fuerza o intensidad de la dependencia de las variables aleatorias dentro de un proceso estocástico, ello mediante el uso de las autocovarianzas. Cuando las covarianzas son normalizadas respecto de la varianza, el resultado es un término que es independiente de las unidad de medida aplicada, y se conoce como la \emph{función de autocorrelación}.

Por su parte, un estimador consistente de la función de autocorrelación estará dado por:
\begin{equation}
    \hat{\rho}(\tau) = \frac{\sum^{T - \tau}_{t=1} (X_t - \hat{\mu})(X_{t+\tau} - \hat{\mu})}{\sum^T_{t=1} (X_t - \hat{\mu})^2} = \frac{\hat{\gamma}(\tau)}{\hat{\gamma}(0)} \mbox{, para } \tau = 1, 2, \ldots, T-1
    \label{eq:EqAutoCorr}
\end{equation}

El estimador de la ecuación \eqref{eq:EqAutoCorr} es asintóticamente insesgado y \textbf{es relevante puesto que nos dice si una serie de tiempo con estacionariedad débil esta serialmente correlacionada si y solo si \(\hat{\rho}(\tau)\neq0\)}.

\hypertarget{ruido-blanco}{%
\subsection{Ruido Blanco}\label{ruido-blanco}}

Supongamos una serie de tiempo denotada por: \(\{U_t\}^T_{t = 0}\). Decimos que el proceso estocástico \(\{U_t\}\) es un \emph{proceso estocástico puramente aleatorio} o es un \emph{proceso estocástico de ruido blanco o caminata aleatoria}, si éste tiene las siguientes propiedades:

\begin{itemize}
\item
  \(\mathbb{E}[U_t] = 0\), \(\forall t\)
\item
  \(Var[U_t] = \mathbb{E}[(U_t - \mu_t)^2] = \mathbb{E}[(U_t - \mu)^2] = \mathbb{E}[(U_t)^2] = \sigma^2\), \(\forall t\)
\item
  \(Cov[U_t,U_s] = \mathbb{E}[(U_t - \mu_t)(U_s - \mu_s)] = \mathbb{E}[(U_t - \mu)(U_s - \mu)] = \mathbb{E}[U_t U_s] = 0\), \(\forall t \neq s\).
\item
  \(\hat{\rho}(\tau)=0\)
\end{itemize}

En palabras. Un proceso \(U_t\) es un ruido blanco si su valor promedio es cero (0), tiene una varianza finita y constante, y además no le importa la historia pasada, así su valor presente no se ve influenciado por sus valores pasados no importando respecto de que periodo se tome referencia.

Para procesos estacionarios, dicha función de autocorrelación esta dada por:
\begin{equation}
    \rho(\tau) = \frac{\mathbb{E}[(X_t - \mu)(X_{t+\tau} - \mu)]}{\mathbb{E}[(X_t - \mu)^2]} = \frac{\gamma(\tau)}{\gamma(0)} 
\end{equation}

\hypertarget{estimaciuxf3n-1}{%
\section{Estimación}\label{estimaciuxf3n-1}}

\hypertarget{dependencias-2}{%
\subsection{Dependencias}\label{dependencias-2}}

\begin{Shaded}
\begin{Highlighting}[]
\CommentTok{\#install.packages("pacman")}
\CommentTok{\#pacman nos permite cargar varias librerias en una sola línea}
\FunctionTok{library}\NormalTok{(pacman)}
\NormalTok{pacman}\SpecialCharTok{::}\FunctionTok{p\_load}\NormalTok{(tidyverse,BatchGetSymbols,ggplot2,lubridate,readxl,forecast,stats)}
\end{Highlighting}
\end{Shaded}

\hypertarget{caminata}{%
\section{Caminata}\label{caminata}}

\begin{Shaded}
\begin{Highlighting}[]

\FunctionTok{set.seed}\NormalTok{(}\DecValTok{1234}\NormalTok{)}
\CommentTok{\# Utilizaremos una función guardada en un archivo a parte}
\CommentTok{\# Llamamos a la función:}
\FunctionTok{source}\NormalTok{(}\StringTok{"funciones/Caminata.R"}\NormalTok{)}

\CommentTok{\# Definimos argumentos de la función}
\NormalTok{Opciones }\OtherTok{\textless{}{-}} \FunctionTok{c}\NormalTok{(}\SpecialCharTok{{-}}\DecValTok{1}\NormalTok{, }\DecValTok{1}\NormalTok{)}
\CommentTok{\#}
\NormalTok{Soporte }\OtherTok{\textless{}{-}} \DecValTok{10000}

\CommentTok{\# Vamos a réplicar el proceso con estos parámetros}
\NormalTok{Rango }\OtherTok{\textless{}{-}} \DecValTok{200}
\CommentTok{\#}
\NormalTok{Caminos }\OtherTok{\textless{}{-}} \DecValTok{10}

\CommentTok{\#}

\ControlFlowTok{for}\NormalTok{(i }\ControlFlowTok{in} \DecValTok{1}\SpecialCharTok{:}\NormalTok{Caminos)\{}
\NormalTok{  TT }\OtherTok{\textless{}{-}} \FunctionTok{data.matrix}\NormalTok{(}\FunctionTok{data.frame}\NormalTok{(}\FunctionTok{Caminata}\NormalTok{(Opciones, Soporte)[}\DecValTok{1}\NormalTok{]))}
  \CommentTok{\#}
\NormalTok{  G\_t }\OtherTok{\textless{}{-}} \FunctionTok{data.matrix}\NormalTok{(}\FunctionTok{data.frame}\NormalTok{(}\FunctionTok{Caminata}\NormalTok{(Opciones, Soporte)[}\DecValTok{2}\NormalTok{]))}
  \CommentTok{\#}
  \FunctionTok{plot}\NormalTok{(TT, G\_t, }\AttributeTok{col =} \StringTok{"blue"}\NormalTok{, }\AttributeTok{type =} \StringTok{"l"}\NormalTok{, }\AttributeTok{ylab =} \StringTok{"Ganancias"}\NormalTok{, }\AttributeTok{xlab =} \StringTok{"Tiempo"}\NormalTok{, }\AttributeTok{ylim =} \FunctionTok{c}\NormalTok{(}\SpecialCharTok{{-}}\NormalTok{Rango,Rango))}
  \CommentTok{\#}
  \FunctionTok{par}\NormalTok{(}\AttributeTok{new =} \ConstantTok{TRUE}\NormalTok{)}
  \CommentTok{\#}
\NormalTok{  i }\OtherTok{\textless{}{-}}\NormalTok{ i }\SpecialCharTok{+}\DecValTok{1}
\NormalTok{\}}
\CommentTok{\#}
\FunctionTok{par}\NormalTok{(}\AttributeTok{new =} \ConstantTok{FALSE}\NormalTok{)}
\end{Highlighting}
\end{Shaded}

\begin{figure}

{\centering \includegraphics{06-ESTACIONARIEDAD_files/figure-latex/Caminata10-1} 

}

\caption{Ejemplo de 10 trayectorias de la caminata aleatoria, cuando sólo es posible cambios de +1 y -1}\label{fig:Caminata10}
\end{figure}

Así, el proceso estocástico dado por la caminata alaeatoria sin un
término de ajuste es estacionario en media, pero no en varianza o en
covarianza, y consecuentemente, en general no estacionario, condición
que contraria al caso del proceso simple descrito en \(U_t\).

Es facil ver que muchas de las posibilidades de realización de este
proceso estocástico (series de tiempo) pueden tomar cualquiera de las
rutas consideradas en el Figura \ref{fig:Caminata10}. Ahora analicemos
un solo camino.

\hypertarget{un-camino}{%
\section{Un camino}\label{un-camino}}

\begin{Shaded}
\begin{Highlighting}[]
\CommentTok{\#Generamos datos}
\NormalTok{  TT1 }\OtherTok{\textless{}{-}} \FunctionTok{data.matrix}\NormalTok{(}\FunctionTok{data.frame}\NormalTok{(}\FunctionTok{Caminata}\NormalTok{(Opciones, Soporte)[}\DecValTok{1}\NormalTok{]))}
\NormalTok{  G\_t1 }\OtherTok{\textless{}{-}} \FunctionTok{data.matrix}\NormalTok{(}\FunctionTok{data.frame}\NormalTok{(}\FunctionTok{Caminata}\NormalTok{(Opciones, Soporte)[}\DecValTok{2}\NormalTok{]))}
\CommentTok{\#Creemos un data frame}
\NormalTok{  dt\_caminata}\OtherTok{\textless{}{-}}\FunctionTok{data.frame}\NormalTok{(TT1,G\_t1)}
  \FunctionTok{colnames}\NormalTok{(dt\_caminata)}\OtherTok{\textless{}{-}}\FunctionTok{c}\NormalTok{(}\StringTok{"t"}\NormalTok{,}\StringTok{"ganancias"}\NormalTok{)}
  \FunctionTok{head}\NormalTok{(dt\_caminata)}
\CommentTok{\#\textgreater{}   t ganancias}
\CommentTok{\#\textgreater{} 1 1        {-}1}
\CommentTok{\#\textgreater{} 2 2        {-}2}
\CommentTok{\#\textgreater{} 3 3        {-}1}
\CommentTok{\#\textgreater{} 4 4        {-}2}
\CommentTok{\#\textgreater{} 5 5        {-}3}
\CommentTok{\#\textgreater{} 6 6        {-}4}
\CommentTok{\#plot}
  \FunctionTok{plot}\NormalTok{(TT1, G\_t1, }\AttributeTok{col =} \StringTok{"blue"}\NormalTok{, }\AttributeTok{type =} \StringTok{"l"}\NormalTok{, }\AttributeTok{ylab =} \StringTok{"Ganancias"}\NormalTok{, }\AttributeTok{xlab =} \StringTok{"Tiempo"}\NormalTok{, }\AttributeTok{ylim =} \FunctionTok{c}\NormalTok{(}\SpecialCharTok{{-}}\NormalTok{Rango,Rango))}
\end{Highlighting}
\end{Shaded}

\begin{figure}
\centering
\includegraphics{06-ESTACIONARIEDAD_files/figure-latex/Caminata1-1.pdf}
\caption{\label{fig:Caminata1}Una Caminata aleatoria cuando sólo es posible cambios de +1 y -1}
\end{figure}

Hay que convertirlo a serie de tiempo

\begin{Shaded}
\begin{Highlighting}[]
\CommentTok{\#serie de tiempo}
\NormalTok{caminata\_ts}\OtherTok{\textless{}{-}}\FunctionTok{ts}\NormalTok{(G\_t1,}\AttributeTok{start=}\DecValTok{1}\NormalTok{,}\AttributeTok{end=}\NormalTok{Soporte)}
\end{Highlighting}
\end{Shaded}

\hypertarget{estacionariedad-caminata}{%
\subsection{Estacionariedad Caminata}\label{estacionariedad-caminata}}

\begin{Shaded}
\begin{Highlighting}[]
\NormalTok{ACF\_caminata\_ts}\OtherTok{\textless{}{-}}\FunctionTok{acf}\NormalTok{(caminata\_ts,}\AttributeTok{na.action =}\NormalTok{ na.pass, }\AttributeTok{main =} \StringTok{"Función de Autocorrelación de una Caminata"}\NormalTok{)}
\end{Highlighting}
\end{Shaded}

\begin{figure}
\centering
\includegraphics{06-ESTACIONARIEDAD_files/figure-latex/ACFCAMINATA1-1.pdf}
\caption{\label{fig:ACFCAMINATA1}Función de Autocorrelación de una Caminata}
\end{figure}

Como se comentó con anterioridad en la Figura \ref{fig:ACFCAMINATA1} es
evidente que la Caminata si tiene autocorrelacion, por lo que nuestro
plot de autocorrelacion tiene valores muy altos en todos los lags.
Veamos los lags.

\begin{Shaded}
\begin{Highlighting}[]
\FunctionTok{gglagplot}\NormalTok{(caminata\_ts,}\AttributeTok{lags=}\DecValTok{10}\NormalTok{,}\AttributeTok{do.lines=}\ConstantTok{FALSE}\NormalTok{,}\AttributeTok{colour=}\ConstantTok{FALSE}\NormalTok{)}\SpecialCharTok{+}\FunctionTok{theme\_light}\NormalTok{()}
\end{Highlighting}
\end{Shaded}

\begin{figure}
\centering
\includegraphics{06-ESTACIONARIEDAD_files/figure-latex/LAGSCAMINATA1-1.pdf}
\caption{\label{fig:LAGSCAMINATA1}Lags de una sola caminata}
\end{figure}

De nuevo, esto al ser creado de manera estandarizada estamos seguros de
que va a ser estacionario en la medio, por lo mismo los lags de la
Figura \ref{fig:LAGSCAMINATA1} se ven tan correlacionados.

\hypertarget{rendimientos-de-un-activo}{%
\section{Rendimientos de un activo}\label{rendimientos-de-un-activo}}

\begin{Shaded}
\begin{Highlighting}[]
\CommentTok{\#Primero determinamos el lapso de tiempo}
\NormalTok{pd}\OtherTok{\textless{}{-}}\FunctionTok{Sys.Date}\NormalTok{()}\SpecialCharTok{{-}}\NormalTok{(}\DecValTok{365}\SpecialCharTok{*}\DecValTok{20}\NormalTok{) }\CommentTok{\#primer fecha}
\NormalTok{pd}
\CommentTok{\#\textgreater{} [1] "2002{-}09{-}26"}
\NormalTok{ld}\OtherTok{\textless{}{-}}\FunctionTok{Sys.Date}\NormalTok{() }\CommentTok{\#última fecha}
\NormalTok{ld}
\CommentTok{\#\textgreater{} [1] "2022{-}09{-}21"}
\CommentTok{\#Intervalos de tiempo}
\NormalTok{int}\OtherTok{\textless{}{-}}\StringTok{"monthly"}
\CommentTok{\#Datos a elegir}
\NormalTok{dt}\OtherTok{\textless{}{-}}\FunctionTok{c}\NormalTok{(}\StringTok{"AMZN"}\NormalTok{)}
\NormalTok{dt2}\OtherTok{\textless{}{-}}\FunctionTok{c}\NormalTok{(}\StringTok{"TSLA"}\NormalTok{)}
\CommentTok{\#Descargando los valores}
\NormalTok{data1}\OtherTok{\textless{}{-}} \FunctionTok{BatchGetSymbols}\NormalTok{(}\AttributeTok{tickers =}\NormalTok{ dt,}
                       \AttributeTok{first.date =}\NormalTok{ pd,}
                       \AttributeTok{last.date =}\NormalTok{ ld,}
                       \AttributeTok{freq.data =}\NormalTok{ int,}
                       \AttributeTok{do.cache =} \ConstantTok{FALSE}\NormalTok{,}
                       \AttributeTok{thresh.bad.data =} \DecValTok{0}\NormalTok{)}
\CommentTok{\#Generando data frame con los valores}
\NormalTok{data\_precio\_amzn}\OtherTok{\textless{}{-}}\NormalTok{data1}\SpecialCharTok{$}\NormalTok{df.tickers}
\FunctionTok{colnames}\NormalTok{(data\_precio\_amzn)}
\CommentTok{\#\textgreater{}  [1] "ticker"              "ref.date"           }
\CommentTok{\#\textgreater{}  [3] "volume"              "price.open"         }
\CommentTok{\#\textgreater{}  [5] "price.high"          "price.low"          }
\CommentTok{\#\textgreater{}  [7] "price.close"         "price.adjusted"     }
\CommentTok{\#\textgreater{}  [9] "ret.adjusted.prices" "ret.closing.prices"}

\CommentTok{\#necesitamos convertir la serie de tiempo de precios en retornos continuos compuestos de los precios de apertura}
\NormalTok{data\_precio\_amzn}\SpecialCharTok{$}\NormalTok{ccrAMZN}\OtherTok{\textless{}{-}}\FunctionTok{c}\NormalTok{(}\ConstantTok{NA}\NormalTok{ ,}\DecValTok{100}\SpecialCharTok{*}\FunctionTok{diff}\NormalTok{(}\FunctionTok{log}\NormalTok{(data\_precio\_amzn}\SpecialCharTok{$}\NormalTok{price.open)))}\CommentTok{\#agregamos un valor NA al principio}
\NormalTok{data\_precio\_amzn}\SpecialCharTok{$}\NormalTok{ccrAMZN}\CommentTok{\#estos son los retornos}
\CommentTok{\#\textgreater{}   [1]           NA   1.86572576  16.90900220  22.83329766}
\CommentTok{\#\textgreater{}   [5] {-}22.98950701  13.39221448   0.95260416  14.27998202}
\CommentTok{\#\textgreater{}   [9]  11.55626991  24.11122449  {-}0.46684144  13.08785513}
\CommentTok{\#\textgreater{}  [13]  11.63599301   3.89974592  12.48104071  {-}0.73260401}
\CommentTok{\#\textgreater{}  [17]  {-}3.06108260  {-}4.08140972 {-}16.32741069   0.99457279}
\CommentTok{\#\textgreater{}  [21]   0.06902105   9.61879412  11.67844638 {-}33.59147712}
\CommentTok{\#\textgreater{}  [25]  {-}0.57381482   7.69997922 {-}18.78100714  15.60691856}
\CommentTok{\#\textgreater{}  [29]  11.66713068  {-}4.43506452 {-}20.41392362  {-}1.23405211}
\CommentTok{\#\textgreater{}  [33]  {-}6.96531282   9.64353557  {-}6.77486160  30.02382912}
\CommentTok{\#\textgreater{}  [37]  {-}5.40177077   6.39945115 {-}12.58398922  20.12391421}
\CommentTok{\#\textgreater{}  [41]  {-}2.92703823  {-}7.77281354 {-}15.93630872  {-}2.10477279}
\CommentTok{\#\textgreater{}  [45]  {-}4.11970307  {-}1.60415929  10.64572285 {-}37.21478392}
\CommentTok{\#\textgreater{}  [49]  15.01070027   3.59739571  17.58906665   5.43570463}
\CommentTok{\#\textgreater{}  [53]  {-}4.00357496  {-}1.90531667   3.54637914   1.33891100}
\CommentTok{\#\textgreater{}  [57]  42.77167396  11.98170332  {-}0.13070948  12.66409736}
\CommentTok{\#\textgreater{}  [61]   2.27857959  15.63296023  {-}6.26135927   2.56510858}
\CommentTok{\#\textgreater{}  [65]   5.74113839 {-}18.78533471 {-}21.72447597  13.78662202}
\CommentTok{\#\textgreater{}  [69]   7.15014819   3.44753666 {-}11.63053849   5.54650897}
\CommentTok{\#\textgreater{}  [73]   8.53074641 {-}14.71605773 {-}24.20236452 {-}29.39126222}
\CommentTok{\#\textgreater{}  [77]  20.09953181  13.15576837   8.77225235  13.27882326}
\CommentTok{\#\textgreater{}  [81]   9.60320128  {-}2.73678724   7.64068237   2.50334747}
\CommentTok{\#\textgreater{}  [85]  {-}6.96036997  13.59745291  24.90536163  14.32806129}
\CommentTok{\#\textgreater{}  [89]  {-}0.50514401 {-}10.08447333  {-}3.70473986  13.45839135}
\CommentTok{\#\textgreater{}  [93]   1.02565002  {-}9.33660068 {-}13.76436786   8.99531736}
\CommentTok{\#\textgreater{}  [97]   5.87517724  21.76202538   4.58513429   8.56726887}
\CommentTok{\#\textgreater{} [101]   1.22598823  {-}6.16865517   1.74979032   4.53458328}
\CommentTok{\#\textgreater{} [105]   7.93222740  {-}0.25978671   4.72685785   9.04110889}
\CommentTok{\#\textgreater{} [109]  {-}4.41608958   0.80039290  {-}4.18766494  {-}8.13529694}
\CommentTok{\#\textgreater{} [113]  {-}8.68550170  {-}1.18960511   3.43828044   9.60224827}
\CommentTok{\#\textgreater{} [117]  14.70991690  {-}9.58159747   9.53799598   2.08880372}
\CommentTok{\#\textgreater{} [121]   5.85976366   2.83140800  {-}8.65274050   7.52661134}
\CommentTok{\#\textgreater{} [125]   1.39202437   4.89612270  {-}2.12710012   1.39936277}
\CommentTok{\#\textgreater{} [129]  {-}5.02332605   5.76221809   3.66491122   8.27850152}
\CommentTok{\#\textgreater{} [133]  {-}6.24554343   9.85520147  15.15285156   8.73395739}
\CommentTok{\#\textgreater{} [137]  {-}0.05013788 {-}10.51934673  {-}0.06687288  {-}5.92858190}
\CommentTok{\#\textgreater{} [141] {-}10.58568315   2.74371861   4.15753522  {-}3.80625428}
\CommentTok{\#\textgreater{} [145]   8.04816141  {-}5.42111518  {-}5.03065911   9.90317538}
\CommentTok{\#\textgreater{} [149]  {-}7.85404256  11.32156221   8.43295383  {-}2.32429615}
\CommentTok{\#\textgreater{} [153]  13.01462014   1.54061721   2.05813986  20.15392877}
\CommentTok{\#\textgreater{} [157]  {-}7.39490376   2.34828935  20.47843365   7.17052347}
\CommentTok{\#\textgreater{} [161]  {-}2.62563883 {-}12.67694180  {-}3.85435390   5.96629237}
\CommentTok{\#\textgreater{} [165]  11.72089295   8.23387458  {-}0.49783030   5.76252931}
\CommentTok{\#\textgreater{} [169]   1.44112456   8.10699776  {-}4.52676184  {-}6.00796092}
\CommentTok{\#\textgreater{} [173]   0.72964776   8.98955770   2.83447462   4.01536256}
\CommentTok{\#\textgreater{} [177]   4.38443827   7.35281333  {-}2.61760725   2.36894617}
\CommentTok{\#\textgreater{} [181]  {-}1.20285851  {-}2.07377928  13.68712240   5.85471090}
\CommentTok{\#\textgreater{} [185]  {-}0.00427124  20.93976645   4.63815978  {-}6.55115525}
\CommentTok{\#\textgreater{} [189]   9.77684681   4.61358018   2.75160333   5.84583306}
\CommentTok{\#\textgreater{} [193]  12.74521370  {-}0.22279328 {-}21.94794383   8.60716489}
\CommentTok{\#\textgreater{} [197] {-}18.86826361  11.20213025   0.98664739   8.39682297}
\CommentTok{\#\textgreater{} [201]   7.12720047  {-}9.38002536   8.85565506  {-}2.70183034}
\CommentTok{\#\textgreater{} [205]  {-}5.58782369  {-}1.36520548   2.37757442   0.91249016}
\CommentTok{\#\textgreater{} [209]   3.83805218   6.98245156  {-}5.31693095   1.37938043}
\CommentTok{\#\textgreater{} [213]  18.97247680   4.64889431  11.92308161  14.25393454}
\CommentTok{\#\textgreater{} [217]   9.27398656  {-}8.41337555  {-}4.66642316   4.05671846}
\CommentTok{\#\textgreater{} [221]   2.52393793  {-}0.84885499  {-}3.59427728  {-}0.31861156}
\CommentTok{\#\textgreater{} [225]  11.12179968  {-}7.17375312   5.72503641  {-}2.40180912}
\CommentTok{\#\textgreater{} [229]   4.18486227  {-}6.11473001   2.18899134   5.30616390}
\CommentTok{\#\textgreater{} [233]  {-}5.62793333 {-}11.06465380   1.80527180   7.20896224}
\CommentTok{\#\textgreater{} [237] {-}29.34750864  {-}0.11853658 {-}13.99459654  23.88072732}
\CommentTok{\#\textgreater{} [241]  {-}6.86965832}
\CommentTok{\#tenemos 20 retornos a lo largo de 20 años}
\end{Highlighting}
\end{Shaded}

Veamos la serie de tiempo

\begin{Shaded}
\begin{Highlighting}[]
\NormalTok{ret\_20\_amazn}\OtherTok{\textless{}{-}}\FunctionTok{ggplot}\NormalTok{(}\AttributeTok{data=}\NormalTok{data\_precio\_amzn, }\FunctionTok{aes}\NormalTok{(}\AttributeTok{x=}\NormalTok{ref.date))}\SpecialCharTok{+}\FunctionTok{geom\_line}\NormalTok{(}\FunctionTok{aes}\NormalTok{(}\AttributeTok{y=}\NormalTok{ccrAMZN))}\SpecialCharTok{+}\FunctionTok{labs}\NormalTok{(}\AttributeTok{title=}\StringTok{"Retornos de AMZN en los últimos 20 años"}\NormalTok{,}\AttributeTok{y=}\StringTok{"Retornos"}\NormalTok{, }\AttributeTok{x=}\StringTok{"Años"}\NormalTok{)}\SpecialCharTok{+}\FunctionTok{theme\_light}\NormalTok{()}
\NormalTok{ret\_20\_amazn}
\end{Highlighting}
\end{Shaded}

\begin{figure}
\centering
\includegraphics{06-ESTACIONARIEDAD_files/figure-latex/amazn20-1.pdf}
\caption{\label{fig:amazn20}Serie de tiempo de los retornos de año en los últimos 20 años}
\end{figure}

\hypertarget{serie-de-tiempo}{%
\subsection{Serie de tiempo}\label{serie-de-tiempo}}

Primero que nada es importante cargar los datos a un objeto series de
tiempo. Esto nos lo permite la función ts(). Además debemos serciorarnos
de que los datos esten en orden cronológico.

\begin{Shaded}
\begin{Highlighting}[]
\NormalTok{data\_precio\_amzn}\OtherTok{\textless{}{-}}\NormalTok{data\_precio\_amzn[}\FunctionTok{order}\NormalTok{(data\_precio\_amzn}\SpecialCharTok{$}\NormalTok{ref.date),]}
\FunctionTok{head}\NormalTok{(data\_precio\_amzn)}\CommentTok{\#dado que ya estaba en orden cronológico nuestro df no cambia}
\CommentTok{\#\textgreater{} \# A tibble: 6 x 11}
\CommentTok{\#\textgreater{}   ticker ref.date     volume price\textasciitilde{}1 price\textasciitilde{}2 price\textasciitilde{}3 price\textasciitilde{}4}
\CommentTok{\#\textgreater{}   \textless{}chr\textgreater{}  \textless{}date\textgreater{}        \textless{}dbl\textgreater{}   \textless{}dbl\textgreater{}   \textless{}dbl\textgreater{}   \textless{}dbl\textgreater{}   \textless{}dbl\textgreater{}}
\CommentTok{\#\textgreater{} 1 AMZN   2002{-}09{-}26   5.58e8   0.796    0.87   0.786   0.796}
\CommentTok{\#\textgreater{} 2 AMZN   2002{-}10{-}01   4.07e9   0.812    1.01   0.800   0.968}
\CommentTok{\#\textgreater{} 3 AMZN   2002{-}11{-}01   4.13e9   0.961    1.23   0.91    1.17 }
\CommentTok{\#\textgreater{} 4 AMZN   2002{-}12{-}02   3.11e9   1.21     1.25   0.922   0.944}
\CommentTok{\#\textgreater{} 5 AMZN   2003{-}01{-}02   3.38e9   0.960    1.16   0.928   1.09 }
\CommentTok{\#\textgreater{} 6 AMZN   2003{-}02{-}03   2.32e9   1.10     1.12   0.980   1.10 }
\CommentTok{\#\textgreater{} \# ... with 4 more variables: price.adjusted \textless{}dbl\textgreater{},}
\CommentTok{\#\textgreater{} \#   ret.adjusted.prices \textless{}dbl\textgreater{}, ret.closing.prices \textless{}dbl\textgreater{},}
\CommentTok{\#\textgreater{} \#   ccrAMZN \textless{}dbl\textgreater{}, and abbreviated variable names}
\CommentTok{\#\textgreater{} \#   1: price.open, 2: price.high, 3: price.low,}
\CommentTok{\#\textgreater{} \#   4: price.close}
\CommentTok{\#hagamos el objeto ts}
\NormalTok{ret\_amazn\_ts}\OtherTok{\textless{}{-}}\FunctionTok{ts}\NormalTok{(data\_precio\_amzn}\SpecialCharTok{$}\NormalTok{ccrAMZN)}
\FunctionTok{plot}\NormalTok{(ret\_amazn\_ts)}\CommentTok{\#de esta manera podemos ver que se cargo bien debido a que es igual al ggplot}
\end{Highlighting}
\end{Shaded}

\includegraphics{06-ESTACIONARIEDAD_files/figure-latex/unnamed-chunk-4-1.pdf}

\hypertarget{estacionariedad-1}{%
\subsection{Estacionariedad}\label{estacionariedad-1}}

\begin{Shaded}
\begin{Highlighting}[]
\CommentTok{\#MA\_m5\textless{}{-}forecast::ma(ret\_amazn\_ts,order=11,centre=TRUE)}
\CommentTok{\#plot(ret\_amazn\_ts)+lines(MA\_m5, col="red", lwd=2)}
\FunctionTok{gglagplot}\NormalTok{(ret\_amazn\_ts,}\AttributeTok{lags=}\DecValTok{20}\NormalTok{,}\AttributeTok{do.lines=}\ConstantTok{FALSE}\NormalTok{,}\AttributeTok{colour=}\ConstantTok{FALSE}\NormalTok{)}\SpecialCharTok{+}\FunctionTok{theme\_light}\NormalTok{()}
\end{Highlighting}
\end{Shaded}

\begin{figure}
\centering
\includegraphics{06-ESTACIONARIEDAD_files/figure-latex/amazn20LAG-1.pdf}
\caption{\label{fig:amazn20LAG}Lag Plot que nos muestra la correlación entre 20 lags}
\end{figure}

\begin{Shaded}
\begin{Highlighting}[]
\NormalTok{ACF\_ret\_amazn\_ts}\OtherTok{\textless{}{-}}\FunctionTok{acf}\NormalTok{(ret\_amazn\_ts,}\AttributeTok{na.action =}\NormalTok{ na.pass)}
\end{Highlighting}
\end{Shaded}

\begin{figure}
\centering
\includegraphics{06-ESTACIONARIEDAD_files/figure-latex/amazn20ACF-1.pdf}
\caption{\label{fig:amazn20ACF}Función de Autocorrelación de los retornos de AMZN en los ultimos 20 años}
\end{figure}

La Figura \ref{fig:amazn20LAG} nos idica la manera en la que se
correlacionan los lags, evidentemente no se puede ver ningún tipo de
correlacioo1ón visible. Similarmente la Figura \ref{fig:amazn20ACF} en
donde se muestra la función de autocorrelación. Expecto al primer lag
--que muestra correlacion debido a que se esta comparando consigo
mismo-- es evidente que no hay correlacioo1ón fuerte entre ninguno de
los lags. Por lo mismo, sería difícil poder encontrar y estimar valores
futuros debido a que la Figura \ref{fig:amazn20LAG} y la Figura
\ref{fig:amazn20ACF} indican que la serie de tiempo de los retornos de
AMZN de la Figura \ref{fig:amazn20} es \textbf{completamente aleatorio y no
hay estacionariedad}.

  \bibliography{book.bib,packages.bib}

\end{document}
